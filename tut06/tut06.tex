\documentclass{beamer}
\usepackage{../tut-slides}
\usepackage{../mathoperatorsAuD}

\usepackage{csquotes}
\usepackage{cancel}

\usepackage{amsmath,amssymb, bm}


\usepackage{booktabs}
\usepackage{tabularx}
\usepackage{tabu}
\newcommand*\head{\rowfont{\bfseries}}
\newcommand*{\tw}{\rowfont{\ttfamily}}

\renewcommand{\tabularxcolumn}[1]{>{\hspace{0pt}}m{#1}}

\usepackage{listings}
\lstdefinestyle{bg}{ 
	basicstyle=\footnotesize\ttfamily,        % the size of the fonts that are used for the code
	breakatwhitespace=false,         % sets if automatic breaks should only happen at whitespace
	breaklines=true,                 % sets automatic line breaking
	commentstyle=\itshape,    	     % comment style
	escapeinside={\%*}{*)},          % if you want to add LaTeX within your code
	extendedchars=true,              % lets you use non-ASCII characters; for 8-bits encodings only, does not work with UTF-8
	firstnumber=1,                % start line enumeration with line 1000
	frame=L,
	backgroundcolor=\color{cdgray!10},
	%	keywordstyle=\bfseries,       % keyword style
	language=Haskell,                 % the language of the code
	numbers=left,                    % where to put the line-numbers; possible: (none, left, right)
	numbersep=8pt,                   % how far the line-numbers are from the code
	numberstyle=\tiny\color{cdgray!50}, % the style that is used for the line-numbers
	rulecolor=\color{cddarkblue}, 
	tabsize=2,	                   % sets default tabsize to 2 spaces
}

\usepackage{textgreek}
\usepackage{tcolorbox}
\usepackage{ragged2e}

\newcommand{\col}[1]{\textcolor{cdpurple}{#1}}
\newcommand{\coll}[1]{\textcolor{cddarkgreen}{#1}}
\newcommand{\step}[2][]{\ensuremath{\overset{{#1} (\text{#2})}{=}}}
\newcommand*{\astep}[2][]{\ensuremath{\overset{{#1} (\text{#2})}&{=}}}

\DeclareMathOperator{\GV}{GV}
\DeclareMathOperator{\FV}{FV}


\begin{document}	
	\title{Programmierung}
	\subtitle{Übung 6: \\ Strukturelle Induktion \& \textlambda-Kalkül (Teil 1)}
	\author{Eric Kunze}
	\email{eric.kunze@tu-dresden.de}
	\city{TU Dresden}
	\date{16. Mai 2022}
%	\institute{Lehrstuhl für Grundlagen der Programmierung}
	\titlegraphic{\includegraphics[width=2cm]{../TUD-white.pdf}}
	
	\maketitle
	

%%%%%%%%%%%%%%%%%%%%%%%%%%%%%%%%%%%%%%%%%%%%%%%%%%%%%%%%%%%%%%%%%%%%%%%%%%%%%

\begin{frame}[fragile] \frametitle{Inhalt}
	\begin{enumerate}
		\item Funktionale Programmierung
		\begin{enumerate}
			\item Einführung in Haskell: Listen
			\item Algebraische Datentypen
			\item Funktionen höherer Ordnung
			\item Typpolymorphie \& Unifikation
			\item Beweis von Programmeigenschaften
			\item \textbf{$\bm\lambda$ -- Kalkül}
		\end{enumerate}
		\item Logikprogrammierung
		\item Implementierung einer imperativen Programmiersprache
		\item Verifikation von Programmeigenschaften
		\item H${}_\text{0}$ -- ein einfacher Kern von Haskell
	\end{enumerate}
\end{frame}



\section{Induktionsbeweise \\ \textit{\normalsize Aufgabe 1}}

\begin{frame} \frametitle{Vollständige Induktion auf $\N$}
	\footnotesize
	
	\textbf{Definition:} natürliche Zahlen $\N \defeq \menge{0, 1, \dots}$
	\begin{equation*}
		\begin{aligned}
			\text{Basisfall: } \quad &&0 &\in \N \\
			\text{Rekursionsfall: }\quad &&x + 1 &\in \N
			\quad \text{ für } \quad x \in \N
		\end{aligned}
	\end{equation*}
	
	\textbf{Beweis von Eigenschaften:}
	Eigenschaft $=$ Prädikat $P$
	\begin{center}
		\fbox{zu zeigen: \hspace{2em} für alle $x \in \N$ gilt $P(x)$}
	\end{center}
	
	\textbf{vollständige Induktion:}
	\begin{itemize}
		\item \textcolor{cdblue}{Induktionsanfang}: \\
		zeige $P(x)$ für $x = 0$
		\item \textcolor{cdblue}{Induktionsvoraussetzung}:\\
		Sei $x \in \N$, sodass $P(x)$ gilt.
		\hfill
		\textcolor{cdgray}{$P(x)$ gilt noch nicht für \textit{alle} $x \in \N$}
		\item \textcolor{cdblue}{Induktionsschritt}: \\
		zeige $P(x+1)$ unter Nutzung der Induktionsvoraussetzung
	\end{itemize}
\end{frame}

\begin{frame} \frametitle{Induktion auf Listen}
	\footnotesize
	
	\textbf{Erinnerung:} Rekursion über Listen \texttt{xs}
	\begin{equation*}
		\begin{aligned}
			\text{Basisfall: } \quad \texttt{xs} &= \texttt{[]} \\
			\text{Rekursionsfall: }\quad \texttt{xs} &= \texttt{(y:ys)} \quad \text{ für } \quad \texttt{ys :: [a]}
		\end{aligned}
	\end{equation*}
	
	\textbf{Beweis von Programmeigenschaften:}
	Eigenschaft $=$ Prädikat $P$
	\begin{center}
		\fbox{zu zeigen: \hspace{2em} für alle \texttt{xs :: [a]} gilt $P(\texttt{xs})$}
	\end{center}
	
	\textbf{Induktion auf Listen:} \vspace{-0.5\baselineskip}
	\begin{itemize}
		\item \textcolor{cdblue}{Induktionsanfang}: \\
		zeige $P(\texttt{xs})$ für \texttt{xs == []}
		\item \textcolor{cdblue}{Induktionsvoraussetzung}:\\
		Sei \texttt{xs :: [a]} eine Liste für die $P(\texttt{xs})$ gilt.
		\item \textcolor{cdblue}{Induktionsschritt}: \\
		zeige $P(\texttt{x:xs})$ für alle \texttt{x :: a} unter Nutzung der Induktionsvoraussetzung
	\end{itemize}
	
	\fbox{\emph{Allgemeiner Hinweis:} Es müssen immer \alert{alle} Variablen quantifiziert werden!}
\end{frame}
	
\begin{frame} \frametitle{Strukturelle Induktion}
	\footnotesize
	
	\textbf{Erinnerung:} Rekursion über Bäume
	\begin{equation*}
		\begin{aligned}
			\text{Basisfall: } \quad &\texttt{Nil} \text{ oder } \texttt{Leaf x} \text{ für } \texttt{x :: a} \\
			\text{Rekursionsfall: }\quad &\texttt{Branch x l r} \text{ für } \texttt{x :: a} \text{ und } \texttt{l,r :: BinTree a}
		\end{aligned}
	\end{equation*}
	
	\begin{center}
		\fbox{zu zeigen: \hspace{2em} für alle \texttt{t :: BinTree a} gilt $P(\texttt{t})$}
	\end{center}
	
	\textbf{strukturelle Induktion:} \vspace{-0.5\baselineskip}
	\begin{itemize}
		\item \textcolor{cdblue}{Induktionsanfang}: \\
		zeige $P(\texttt{t})$ für \texttt{t == Nil} oder \texttt{t == Leaf x} für alle \texttt{x :: a}
		\item \textcolor{cdblue}{Induktionsvoraussetzung}:\\
		Seien \texttt{l, r :: BinTree a} zwei Bäume, sodass $P(\texttt{l})$ und $P(\texttt{r})$ gilt.
		\item \textcolor{cdblue}{Induktionsschritt}: \\
		zeige $P(\texttt{Branch x l r})$ für alle \texttt{x :: a} unter Nutzung der Induktionsvoraussetzung
	\end{itemize}
	
	\fbox{\emph{Allgemeiner Hinweis:} Es müssen immer \alert{alle} Variablen quantifiziert werden!}
\end{frame}


\begin{frame}[fragile] \frametitle{Aufgabe 1}
	\footnotesize
	\begin{lstlisting}[style=bg, basicstyle=\ttfamily\scriptsize]
data Tree a = Node a (Tree a) (Tree a) | Leaf a

mirror :: Tree a -> Tree a
mirror (Node x t1 t2) = Node x (mirror t2) (mirror t1)
mirror (Leaf x) = Leaf x

yield :: Tree a -> [a]
yield (Node _ t1 t2) = yield t1 ++ yield t2
yield (Leaf x) = [x]
	\end{lstlisting}
	verwendbare Eigenschaften:
	\begin{align*}
		\texttt{reverse [x]} &\texttt{ = }  \texttt{[x]} \tag{E1} \\
		\texttt{reverse (xs ++ ys)} &\texttt{ = } \texttt{reverse ys ++ reverse xs} \tag{E2}
	\end{align*}

	Mittels struktureller Induktion ist folgende Aussage zu zeigen:
	\begin{tcolorbox}[colback=cddarkblue!10, boxrule=0pt, arc=0mm]
		\textit{Für jeden Typ \texttt{a}} und jeden Baum \texttt{t :: Tree a} gilt 
		\begin{equation*}
			\texttt{reverse (yield t) = yield (mirror t)} 
		\end{equation*}
	\end{tcolorbox}
\end{frame}
	
\begin{frame} \frametitle{Aufgabe 1}
	\scriptsize
	\textbf{Induktionsanfang:} Sei \texttt{a} ein Typ und \texttt{x :: a} beliebig sowie \texttt{t = Leaf x}. 
	\begin{align*}
			\text{linke Seite: } \quad &\texttt{reverse (yield (Leaf x))} \step{9} \texttt{reverse [x]} \step{E1} \texttt{[x]} \\
			\text{rechte Seite: } \quad & \texttt{yield (mirror (Leaf x))} \step{5} \texttt{yield (Leaf x)} \step{9} \texttt{[x]}
		\end{align*}
	
	\textbf{Induktionsvoraussetzung:} Seien \texttt{a} ein Typ und \texttt{l, r:: Tree a}, sodass
	\begin{align}
			\texttt{reverse (yield r) = yield (mirror l)}  \tag{IV1} \\
			\texttt{reverse (yield r) = yield (mirror r)}  \tag{IV2}
	\end{align}
	gilt.
\end{frame}

\begin{frame} \frametitle{Aufgabe 1}
	\scriptsize
	\textbf{Induktionsschritt:} Sei \texttt{a} ein Typ und \texttt{x :: a} beliebig. Es gilt
	\begin{align*}
		&\texttt{reverse (yield (Node x l r))} \\
		\astep{8} \texttt{reverse (yield l ++ yield r)} \\
		\astep{E2}\texttt{reverse (yield r) ++ reverse (yield l)} \\
		\astep{IV1} \texttt{reverse (yield r) ++ yield (mirror l)} \\
		\astep{IV2} \texttt{yield (mirror r) ++ yield (mirror l)} \\
		\astep{8} \texttt{yield (Node x (mirror r) (mirror l))} \\
		\astep{4} \texttt{yield (mirror (Node x l r))}
		\end{align*}
\end{frame}

\section{Der \textlambda-Kalkül \\ \textit{\normalsize Aufgaben 2 \& 3}}

\begin{frame} \frametitle{$\lambda$-Kalkül}
	\footnotesize
	\begin{itemize}
		\item weitere funktionale Programmiersprache
		\item Programme = $\lambda$-Terme
		\item Vorstellung: \textit{anonyme} Funktionen
	\end{itemize}
	\pause
	
	Sei $X$ eine Menge mit Variablen, $\Sigma$ eine Menge mit Symbolen. Die gültigen $\lambda$-Terme sind \textit{induktiv} definiert:
	\begin{enumerate}[<+->]
		\item \textbf{Atome} (Variablen oder Symbole) sind gültige $\lambda$-Terme.
		\item \textbf{Abstraktion}: Ist $t$ ein gültiger $\lambda$-Term und $x \in X$ eine Variable, dann ist auch $(\lambda x . t)$ ein gültiger $\lambda$-Term.
		\item \textbf{Applikation}: Sind $t_1$ und $t_2$ gültige $\lambda$-Terme, dann ist auch $(t_1 \ t_2)$ ein gültiger $\lambda$-Term.
	\end{enumerate}
\end{frame}

\begin{frame} \frametitle{Beispiele (informell)}
	\footnotesize
	Vorstellung: Jeder $\lambda$-Term beschreibt eine anonyme Funktion. \pause
	\begin{itemize}[<+->]
		\item Abstraktion gibt das Argument an: 
		\begin{equation*}
		\lambda x . t \quad \leftrightarrow \quad f(x) = t
		\end{equation*}
		\item Applikation beschreibt Funktionsanwendung (Einsetzen): 
		\begin{equation*}
			((\lambda x . t) \ 2) \quad \leftrightarrow \quad f(2)
		\end{equation*}
	\end{itemize}

	\pause
	
	\textbf{Beispiel:}
	\begin{align*}
		\text{quadriere } &= \lambda x . \ x*x \\
		((\lambda x . x*x) \ 2) &= 2 * 2 = 4
	\end{align*}
	$\nearrow \beta$-Reduktion 
\end{frame}

\begin{frame} \frametitle{Verabredungen}
	\footnotesize
	\begin{itemize}[<+->]
		\item Applikation ist linksassoziativ:
		\begin{equation*}
			((t_1 \ t_2) \ t_3) = t_1 \ t_2 \ t_3 
		\end{equation*}
		\item mehrfache Abstraktion:
		\begin{equation*}
			(\lambda x_1 . (\lambda x_2 . (\lambda x_3 . t))) = \lambda x_1 x_2 x_3 . t 
		\end{equation*}
		\item Applikation vor Abstraktion:
		\begin{align*}
			(\lambda x . \ x \ y) &= (\lambda x . (x \ y)) \\
			&\neq ((\lambda x . x) \ y)
		\end{align*}
	\end{itemize}
\end{frame}

\begin{frame} \frametitle{Gebundene und freie Vorkommen}
	\footnotesize
	Mengen $\FV(t)$ und $\GV(t)$ geben frei bzw. gebunden \textit{vorkommende} Variablen von $t$ an --- induktive Definition
	
	\pause
	
	\begin{itemize}[<+->]
		\item einzelne \textbf{Variablen} sind immer frei: \\
		$x \in X \follows \FV(x) = \menge{x} , \GV(x) = \emptyset$
		\item \textbf{Symbole} sind weder frei noch gebunden
		\item \textbf{Applikation}: Sei $t = (t_1 \  t_2)$. Dann \\
		$\follows 
		\FV(t) = \FV(t_1) \cup \FV(t_2) , \enskip
		\GV(t) = \GV(t_1) \cup \GV(t_2)$
		\item \textbf{Abstraktion}: $t = \lambda x . t'$ \\
		$\follows 
		\FV(t) = \FV(t') \setminus \menge{x} , \enskip
		\GV(t) = \GV(t') \cup \menge{x}$
	\end{itemize}
\end{frame}

\begin{frame} \frametitle{$\beta$ -- Reduktion}
	\footnotesize
	\begin{block}{\textbeta--Reduktion}
		Seien $s,t \in \lambda(\Sigma)$ gültige $\lambda$-Terme und es gilt $\GV(\alert<2->{t}) \cap \FV(s) = \emptyset$. 
		\begin{equation*}
			(\lambda x . \alert<2->{t}) \ s \quad \longrightarrow_\beta \quad t[x / s]
		\end{equation*}
	\end{block}
	
	\pause
	
	\begin{itemize}
		\item Bedeutung von $t[x / s]$: Ersetze jedes \textit{freie} Vorkommen von $x$ in $t$ durch $s$. 
		\item Erinnerung: Vorstellung der Applikation als \enquote{Einsetzen} in Funktionen
		\item beachte: Abstraktion $\lambda x$ entfällt
	\end{itemize}

	\pause
	
	\textbf{Bsp.:} Seien die Symbole gegeben durch $\Sigma = \menge{3,a}$.
	\begin{equation*}
		(\lambda x . \underbrace{+ x 3}_{\GV = \emptyset}) \ (\underbrace{\lambda z . a}_{\FV = \emptyset}) \quad \longrightarrow_\beta \quad + \ (\lambda z . a) 3
	\end{equation*}
\end{frame}

\begin{frame} \frametitle{$\alpha$ -- Konversion }
	\footnotesize
	\begin{itemize}
		\item Was machen wir, wenn Voraussetzung $\FV(t) \cap \GV(t) = \emptyset$ für \textbeta-Reduktion nicht erfüllt ist?
		\item einfacher Ausweg: entsprechende Variablen umbenennen, sodass Bedingung erfüllt ist
	\end{itemize}

	\pause
	
	\begin{block}{\textalpha --Konversion}
		Sei $t \in \lambda(\Sigma)$ und $z \notin \GV(t) \cup \FV(t)$.
		\begin{equation*}
			(\lambda x . t) \quad \longrightarrow_\alpha \quad \lambda z . t[x / z]
		\end{equation*}
	\end{block}

	\pause
	
	\textbf{Bsp.:} Seien die Symbole gegeben durch $\Sigma = \menge{3,a}$.
	\begin{equation*}
		(\lambda x . (\underbrace{\lambda y . + x y}_{\GV = \menge{y}})) \ (\underbrace{y}_{\FV = \menge{y}}) 
		\quad \longrightarrow_\alpha \quad 
		(\lambda x . (\underbrace{\lambda z . + x z}_{\GV = \menge{z}})) \ (\underbrace{y}_{\FV = \menge{y}})
	\end{equation*}
\end{frame}

\begin{frame} \frametitle{Normalform \hfill \alert{\small NEW}}
	\footnotesize \justifying
	Nutzt man sowohl $\alpha$-Konversionen $\Rightarrow_\alpha$ als auch $\beta$-Reduktionen $\Rightarrow_\beta$, so spricht man von der \textit{Rechenvorschrift} $\Rightarrow$ des $\lambda$-Kalküls.
	
	Führt man mehrere Schritte direkt aus, so schreibt man $\Rightarrow^\ast$ statt $\Rightarrow$.
	\pause
	
	Wenn $t \Rightarrow^\ast s$ und es gibt keine $\lambda$-Terme $s_1, s_2$ mit $s \Rightarrow_\alpha^\ast s_1 \Rightarrow_\beta^\ast s_2$, dann heißt $s$ ($\beta$-)\textbf{Normalform} von $t$.
	
	\textbf{Anschauung:} \vspace{-.7\baselineskip}
	\begin{itemize}
		\item möglichst einfache Form eines Lambda-Terms
		\item \enquote{fertig ausgerechnete} Funktion
	\end{itemize}
	\pause
	
	Die Rechenvorschrift $\Rightarrow$ ist \textit{konfluent}, d. h. für alle $\lambda$-Terme $t, t_1 , t_2$ gilt: wenn $t \Rightarrow^\ast t_1$ und $t \Rightarrow^\ast t_2$, dann gibt es einen $\lambda$-Term $s$ mit $t_1 \Rightarrow^\ast s$ und $t_2 \Rightarrow^\ast s$.
	
	Damit gilt auch: wenn eine Normalform existiert, dann ist sie \textit{eindeutig} (egal welche Schritte man zwischendurch auswählt).
\end{frame}

\begin{frame} \frametitle{Normalform --- Vorgehen \hfill \alert{\small NEW}}
	\footnotesize
	\textbf{Vorgehen:}
	\begin{enumerate}
		\item Bestimmung von Stellen, an denen reduziert werden kann \\
		Anforderungen: (Teil-)Term der Form $\big( \underbrace{(\lambda x . t)}_{= t'} s \big)$
		\begin{itemize} \footnotesize
			\item Applikation $(t' s)$: es muss ein Term $s$ zum Einsetzen vorhanden sein
			\item Abstraktion in $t' = (\lambda x . t)$: der erste Term braucht eine \enquote{Variable} $x$, für die etwas eingesetzt werden kann
		\end{itemize} 
		\textit{Achtung: implizite Klammerung (Linksassoziativität) beachten!}
		\pause
		\item Bestimmung gebundener und freier Vorkommen: $\big( (\lambda x . \underbrace{t}_{\GV}) \underbrace{s}_{\FV} \big)$
		\pause
		\item Falls $\GV(t) \cap \FV(s) \neq \emptyset$: $\qquad \alpha$-Konversion
		\begin{itemize} \footnotesize
			\item Umbenennung der gebunden Vorkommen in $t$
		\end{itemize}
		\item Falls $\GV(t) \cap \FV(s) = \emptyset$: $\qquad \beta$-Reduktion
		\begin{itemize} \footnotesize
			\item Streichen der Abstraktion $\lambda x.$
			\item Setze für jedes Vorkommen von $x$ in $t$ den Term $s$ sein
		\end{itemize}
	\end{enumerate}
\end{frame}

\newcommand{\acol}[2]{\textcolor<#2->{cdpurple}{#1}}
\newcommand{\acoll}[2]{\textcolor<#2->{cddarkgreen}{#1}}
\begin{frame} \frametitle{Aufgabe 2 --- Teil (a)}
	
	\begin{itemize}[<+->]
		\item $t_1 = (\lambda \acoll{x}{2} . \acoll{x}{2} \acol{y}{3}) \ (\lambda \acoll{y}{2} . \acoll{y}{2})$:
			\begin{itemize}
				\item $\GV(t_1) = \menge{\coll{x},\coll{y}}$
				\item $\FV(t_1) = \menge{\col{y}}$
			\end{itemize}
		\item $t_2 = (\lambda \acoll{x}{5} . (\lambda \acoll{y}{5} . \acol{z}{6} (\lambda \acoll{z}{5} . \acoll{z}{5} (\lambda \acoll{x}{5} . \acoll{y}{5}))))$
			\begin{itemize}
				\item $\GV(t_2) = \menge{\coll{x},\coll{y},\coll{z}}$
				\item $\FV(t_2) = \menge{\col{z}}$
			\end{itemize}
		\item $t_3 = (\lambda \acoll{x}{8} . (\lambda \acoll{y}{8} . \acol{z}{9} (\acoll{y}{8} \acol{z}{9})))(\lambda \acoll{x}{8} . \acol{y}{9} (\lambda \acoll{y}{8}. \acoll{y}{8} ))$
			\begin{itemize}
				\item $\GV(t_3) = \menge{\coll{x},\coll{y}}$
				\item $\FV(t_3) = \menge{\col{y},\col{z}}$
			\end{itemize}
	\end{itemize}
\end{frame}

\begin{frame}[t] \frametitle{Aufgabe 2 --- Teil (b)}
	Term 1
	\begin{align*}
		&(\lambda x . (\underbrace{\lambda y . x \ z \ (y \ z))}_{\GV=\menge{y}}) \ \underbrace{(\lambda x . y \ (\lambda y . y))}_{\FV = \menge{y}}
		\\
		\Rightarrow_\alpha \quad
		& (\lambda x . (\underbrace{\lambda y_1 . x \ z \ (y_1 \ z))}_{\GV=\menge{y}}) \ \underbrace{(\lambda x . y \ (\lambda y . y))}_{\FV = \menge{y}}
		\\
		\Rightarrow_\beta \quad
		& (\lambda y_1 . (\lambda x . \underbrace{y \ (\lambda y . y)}_{\GV= \menge{y}}) \ \underbrace{z}_{\FV = \menge{z}} \ (y_1 \ z))
		\\
		\Rightarrow_\beta \quad
		& (\lambda y_1 . (y \ (\lambda y . y)) \ (y_1 \ z))
		\\
		= \ & (\lambda y_1 . y \ (\lambda y . y) \ (y_1 \ z))
	\end{align*}
\end{frame}


\begin{frame}[t] \frametitle{Aufgabe 2 --- Teil (b)}
	 Term 2
	\begin{align*}
		& (\lambda x . (\underbrace{\lambda y . (\lambda z . z)}_{\GV=\menge{y,z}})) \ \underbrace{x}_{\FV = \menge{x}} ( + \ y \ 1)
		\\
		\Rightarrow_\beta \quad
		& (\lambda y . \underbrace{(\lambda z . z)}_{\GV=\menge{z}}) \ \underbrace{( + \ y \ 1)}_{\FV = \menge{y}}
		\\
		\Rightarrow_\beta \quad
		& (\lambda z . z)
	\end{align*}
\end{frame}


\begin{frame}[t] \frametitle{Aufgabe 2 --- Teil (b)}
	\small
	Term 3
	\begin{align*}
		& (\lambda x . (\lambda y . x \ (\lambda z . y \ z))) \ (((\lambda x . \underbrace{(\lambda y . y )}_{\GV= \menge{y}}) \ \underbrace{8}_{\FV = \emptyset} ) \ (\lambda x . (\lambda y . y) \ x ))
		\\
		\Rightarrow_\beta \quad
		& (\lambda x . (\lambda y . x \ (\lambda z . y \ z))) \ ((\lambda y . y )) \ (\lambda x . (\lambda y . y) \ x ))
		\\
		\Rightarrow_\beta \quad
		& (\lambda x . (\lambda y . x \ (\lambda z . y \ z))) \ ((\lambda y . y ) \ (\lambda x . (\lambda y . \underbrace{y}_{\GV=\emptyset}) \ \underbrace{x}_{\menge{x}} ))
		\\
		\Rightarrow_\beta \quad
		& (\lambda x . (\lambda y . x \ (\lambda z . y \ z))) \ ((\lambda y . \underbrace{y}_{\GV = \emptyset} ) \ \underbrace{(\lambda x . x)}_{\FV = \emptyset})
		\\
		\Rightarrow_\beta \quad
		& (\lambda x . \underbrace{(\lambda y . x \ (\lambda z . y \ z))}_{\GV = \menge{y,z}}) \ \underbrace{(\lambda x . x)}_{\FV = \emptyset}
		\\
		\Rightarrow_\beta \quad
		& (\lambda y . (\lambda x . \underbrace{x}_{\GV = \emptyset}) \ \underbrace{(\lambda z . y \ z)}_{\FV = \menge{y}})
		\\
		\Rightarrow_\beta \quad
		& (\lambda y . (\lambda z . y \ z)) 
		\quad = \quad
		(\lambda yz . y \ z) 
	\end{align*}
\end{frame}

\begin{frame}[t] \frametitle{Aufgabe 2 --- Teil (b)}
	\small
	Term 4
	\begin{align*}
		& (\lambda h . (\lambda x . h \ (x \ x)) \ (\lambda x . h \ (x \ x))) ((\lambda x . \underbrace{x}_{\GV = \emptyset}) \ \underbrace{(+ \ 1 \ 5)}_{ \FV = \emptyset})
		\\
		\Rightarrow_\beta \quad
		& (\lambda h . (\lambda x . \underbrace{h \ (x \ x)}_{\GV = \emptyset}) \ \underbrace{(\lambda x . h \ (x \ x))}_{\FV = \menge{h}}) (+ \ 1 \ 5)
		\\
		\Rightarrow_\beta \quad
		& (\lambda h . h \ ((\lambda x . \underbrace{h \ (x \ x)}_{\GV = \emptyset}) \ \underbrace{(\lambda x . h \ (x \ x))}_{\FV= \menge{h}})) \ (+ \ 1 \ 5)
		\\
		\Rightarrow_\beta \quad
		& (\lambda h . h \ ( h \ ((\lambda x . \underbrace{h \ (x \ x)}_{\GV = \emptyset}) \ \underbrace{(\lambda x . h \ (x \ x))}_{\FV= \menge{h}}))) \ (+ \ 1 \ 5)
		\\
		\longrightarrow \quad
		&\text{endlose Rekursion, bei der $h$ durch $(+ \ 1 \ 5 )$ noch} \\
		&\text{reduziert werden könnte}
		\\
		\Rightarrow_\beta \quad
		& (+ \ 1 \ 5) \ ( (+ \ 1 \ 5) \ ((\lambda x . (+ \ 1 \ 5 ) \ (x \ x)) \ (\lambda x . (+ \ 1 \ 5 ) \ (x \ x))))	
	\end{align*}
\end{frame}

\begin{frame}[t] \frametitle{Aufgabe 2 --- Teil (b)}
	Term 5
	\begin{align*}
		& (\lambda f . \underbrace{(\lambda a . (\lambda b . f \ a \ b))}_{\GV = \menge{a,b}}) \ \underbrace{(\lambda x . (\lambda y . x ))}_{\FV= \emptyset}) 
		\\
		\Rightarrow_\beta \quad
		& (\lambda a . (\lambda b . (\lambda x . \underbrace{(\lambda y . x )}_{\GV = \menge{y}}) \ \underbrace{a}_{\FV = \menge{a}} \ b)) 
		\\
		\Rightarrow_\beta \quad
		& (\lambda a . (\lambda b . (\lambda y . \underbrace{a}_{\GV = \emptyset}) \ \underbrace{b}_{\FV = \menge{b}})) 
		\\
		\Rightarrow_\beta \quad
		& (\lambda a . (\lambda b . a)) 
		\\
		= \ & (\lambda ab . a)
	\end{align*}
\end{frame}

\begin{frame} \frametitle{Aufgabe 3}
	\begin{tabbing}
		\emph{(a)} $\quad$ \= $A$ mit $A \ t \ s \ u \Rightarrow^\ast s$:
		\pause $\qquad$ \= $A = (\lambda xyz \ . \ y)$ \\[6pt]
		\pause
		\emph{(b)} \> $B$ mit $B \ t \ s \Rightarrow^\ast s \ t$: 
		\pause \> $B = (\lambda xy \ . \ yx)$ \\[6pt]
		\pause
		\emph{(c)} \> $C$ mit $C \ C \Rightarrow^\ast C \ C$: 
		\pause \> $C =( \lambda x \ . \ xx)$  \\
		\pause \> denn: 
		$(\lambda x . \underbrace{xx}_{GV=\emptyset}) \underbrace{(\lambda x.xx)}_{FV = \emptyset} \quad \Rightarrow^{\beta} \quad (\lambda x . xx) (\lambda x.xx)$ \\[6pt]
		\pause
		\emph{(d)} \> $D$ mit $D \Rightarrow^\ast D$: 
		\pause \> $D = (C \ C)$ \\[6pt]
		\pause	
		\emph{(e)} \> $E$ mit $E \ E \ t \Rightarrow^\ast E \ t \ E$: 
		\pause \> $E = (\lambda xy \ . \ xyx)$ \\ \pause
		\> denn:
	\end{tabbing}
	\small
	\begin{align*}
		(\lambda x\underbrace{y \ . \ xyx}_{GV = \{y\}}) (\underbrace{\lambda xy \ . \ xyx}_{FV = \emptyset}) \ t \enskip
		\onslide<13->{&\Rightarrow^{\beta} \enskip%
		(\lambda y \ . \ (\lambda xy \ . \ xyx) \ y \ (\lambda xy \ . \ xyx)) \ t } \\
		\onslide<14->{&\Rightarrow^{\beta} \enskip %
		(\underbrace{\lambda xy \ . \ xyx}_{ = E}) \ t \ (\underbrace{\lambda xy \ . \ xyx}_{=E})}
	\end{align*}
\end{frame}

\end{document}

