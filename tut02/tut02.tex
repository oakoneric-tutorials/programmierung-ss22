\documentclass{beamer}
\usepackage{../tut-slides}
\usepackage{../mathoperatorsAuD}

\usepackage{csquotes}
\usepackage{cancel}

\usepackage{amsmath,amssymb}
\usepackage{enumerate}
\usepackage[inline]{enumitem} 		%customize label
\newcommand{\labelitemi}{\raisebox{1pt}{\scalebox{.9}{$\blacktriangleright$}}}
\newcommand{\labelitemii}{$\vartriangleright$}
\newcommand{\labelitemiii}{--}

\usepackage{booktabs}
\usepackage{tabularx}
\usepackage{tabu}
\newcommand*\head{\rowfont{\bfseries}}
\newcommand*{\tw}{\rowfont{\ttfamily}}

\renewcommand{\tabularxcolumn}[1]{>{\hspace{0pt}}m{#1}}

\usepackage[skins]{tcolorbox}
\tcbuselibrary{listings}
\newtcblisting{codebox}{%
	top=0pt,
	bottom=0pt,
	left=0pt,
	right=0pt,
	boxsep=5pt,
	enhanced,clip upper,
	colback=white,
	colframe=cddarkblue,
	fonttitle=\bfseries,
	listing only}
\newtcblisting{smallcodebox}{%
	top=0pt,
	bottom=0pt,
	left=0pt,
	right=0pt,
	boxsep=5pt,
	enhanced,clip upper,
	colback=white,
	colframe=cddarkblue,
	fonttitle=\bfseries,
	fontupper=\tiny,
	listing only}

\newtcblisting{commandshell}{%
	colback=black,colupper=white,colframe=yellow!75!black,
	listing only,listing options={style=tcblatex,language=sh},
	every listing line={\textcolor{red}{\small\ttfamily\bfseries root \$> }}}

\begin{document}	
	\title{Programmierung}
	\subtitle{Übung 2: Listen, Zeichenketten \& Bäume}
	\author{Eric Kunze}
	\email{eric.kunze@mailbox.tu-dresden.de}
%	\city{TU Dresden}
	\date{}
%	\institute{Lehrstuhl für Grundlagen der Programmierung}
%	\titlegraphic{\includegraphics[width=2cm]{../TUD-white.pdf}}
	
	\maketitle


%%%%%%%%%%%%%%%%%%%%%%%%%%%%%%%%%%%%%%%%%%%%%%%%%%%%%%%%%%%%%%%%%%%%%%%%%%%%%

\section{Übungsblatt 1 \\ \textit{\normalsize Zusatzaufgabe}}

\begin{frame} \frametitle{Übungsblatt 1 -- Zusatzaufgabe}
	\footnotesize
	\textbf{Ziel:} Anzahl der vollständigen Binärbäume mit $n$ Knoten
	
	\textbf{Idee:} Wie erhalten wir volle Binärbäume? --- Ein voller Binärbaum ist
	\begin{itemize}
		\item entweder ein Blatt
		\item oder er besteht aus einer Wurzel und \textit{zwei} Kindern 
	\end{itemize}

	\textbf{Umsetzung:}  
	\begin{itemize}
		\item Rekursionsfall: $n \ge 3$ Knoten 
		\begin{itemize}
			\item ein Wurzelknoten
			\item $n-1$ Knoten für linken und rechten Teilbaum \\ (systematisch alle Möglichkeiten durchlaufen)
		\end{itemize}
		\item Basisfall: 
		\begin{itemize}
			\item $n = 0$: es gibt keinen Baum mit keinen Knoten
			\item $n = 1$: Baum mit einem Knoten $=$ Blatt (davon gibt es genau einen)
		\end{itemize}
	\end{itemize}
\end{frame}

\begin{frame}[fragile] \frametitle{Übungsblatt 1 -- Zusatzaufgabe}
	\footnotesize
	\begin{codebox}
countBinTrees :: Int -> Int
countBinTrees 0 = 0
countBinTrees 1 = 1
countBinTrees n = go (n-1)
  where
    go 0 = 0
    go m = go (m-1) + countBinTrees (n - 1 - m) * 
                      countBinTrees m
	\end{codebox}

	\textbf{Hinweis:} \texttt{go} durchläuft alle Möglichkeiten $n-1$ Knoten so auf zwei (Kind-)Bäume zu verteilen, dass der linke Teilbaum $m$ Knoten und der rechte Teilbaum die übrigen $n - 1 - m$ Knoten besitzt.
\end{frame}


\section{Aufgabe 1 \\ \textit{\normalsize Listen}}
	%
	\begin{frame}[fragile] \frametitle{Listen}
		\begin{description}
			\item[Listen] Wenn \texttt{a} ein Typ ist, dann bezeichnet \texttt{[a]} den Typ ``Liste mit Elementen vom Typ \texttt{a}'', insbesondere haben alle Elemente einer Liste den gleichen Typ
			\bigskip \pause
			\item[cons-Operator `` \texttt{:} '']  ~\\ 
			Trennung von \textit{head} und \textit{tail} einer Liste \\
			\texttt{[x1, x2, x3, x4, x5] = x1 : [x2, x3, x4, x5]}
			\bigskip \pause
			\item[Verkettungsoperator `` \texttt{++} ''] ~\\ 
			Verkettung zweier Listen gleichen Typs \\
			\texttt{[x1, x2] ++ [x3, x4, x5] = [x1, x2, x3, x4, x5]}
		\end{description}
	\end{frame}


\begin{frame}[t, fragile] \frametitle{Rekursion auf Listen --- Beispiel}
	\textbf{Multiplikation einer Liste}
	
	\texttt{prod :: [Int] -> Int}
	
	\pause \bigskip
	
	\begin{codebox}
prod :: [Int] -> Int
prod []     = 1
prod (x:xs) = x * prod xs
	\end{codebox}
\end{frame}

\begin{frame}[t, fragile] \frametitle{Aufgabe 1 -- Teil (a)}
	\textbf{Umkehrung einer Liste}
	
	\texttt{rev :: [Int] -> [Int]}
	
	\pause \bigskip
	
	\begin{codebox}
rev :: [Int] -> [Int]
rev []     = []
rev (x:xs) = rev xs ++ [x]
	\end{codebox}

	\begin{alertblock}{WICHTIG}
		\begin{itemize}
			\item \texttt{Element : [Liste]}
			\item \texttt{[Liste] ++ [Liste]}
		\end{itemize}
	\end{alertblock}
\end{frame}


\begin{frame}[t, fragile] \frametitle{Aufgabe 1 -- Teil (b)}
	\small 
	\textbf{Sortierung einer Liste prüfen}
	
	\texttt{isOrd :: [Int] -> Bool}
	
	\pause \bigskip
	\begin{codebox}
isOrd :: [Int] -> Bool
isOrd []  = True
isOrd [x] = True
isOrd (x:y:xs)
	| x <= y    = isOrd (y:xs)
	| otherwise = False
	\end{codebox}
	\pause
	\begin{codebox}
isOrd' :: [Int] -> Bool
isOrd' []       = True
isOrd' [x]      = True
isOrd' (x:y:xs) = x <= y && isOrd' (y:xs)
	\end{codebox}
\end{frame}

\begin{frame}[t, fragile] \frametitle{Aufgabe 1 -- Teil (c)}
	\small
	\textbf{sortiertes Zusammenfügen zweier (sortierten) Listen}
	
	\texttt{merge :: [Int] -> [Int] -> [Int]}
	
	\pause \bigskip
	\begin{codebox}
merge :: [Int] -> [Int] -> [Int]
merge [] ys = ys
merge xs [] = xs
merge (x:xs) (y:ys)
	| x < y     = x : merge xs (y:ys)
	| otherwise = y : merge (x:xs) ys
	\end{codebox}
	\pause
	Wir können Listen auch ``benennen'' --- Rekursionsfall:
	\begin{codebox}
merge xxs@(x:xs) yys@(y:ys)
	| x < y     = x : merge xs yys
	| otherwise = y : merge xxs ys
	\end{codebox}
\end{frame}

\begin{frame}[t, fragile] \frametitle{Aufgabe 1 -- Teil (d)}
	\small
	\textbf{(unendliche) Liste der Fibonacci-Zahlen}
	
	\texttt{fibs :: [Int]}
	
	\pause \bigskip
	\begin{codebox}
fib :: Int -> Int
fib 0 = 1
fib 1 = 1
fib n = fib (n-1) + fib (n-2)

fibs :: [Int]
fibs = fibAppend 0
	where fibAppend x = fib x : fibAppend (x+1)
	\end{codebox}
	\pause
	\begin{codebox}
fibs :: [Int]
fibs = fibs' 0 1
	where fibs' n m = n : fibs' m (n+m)
	\end{codebox}
\end{frame}

%%%%%%%%%%%%%%%%%%%%%%%%%%%%%%%%%%%%%%%%%%%%%%%%%%%%%%%%%%%%%%%%%%%%%%%%%%%%%%%%%%%%

\section{Aufgabe 2 \\ \textit{\normalsize Zeichen \& Zeichenketten}}

	\begin{frame}[fragile] \frametitle{Zeichen \& Zeichenketten}
		\textbf{Zeichen}
		\begin{itemize}
			\item Datentyp \texttt{Char}
			\item Eingabe in einfachen Anführungszeichen
			\item z.B. \texttt{'a'}, \texttt{'e'}, \texttt{'3'}
		\end{itemize}
		\pause
		\textbf{Zeichenketten}
		\begin{itemize}
			\item Datentyp \texttt{String = [Char]}
			\item Eingabe in doppelten Anführungszeichen
			\item z.B. \texttt{"hallo"}, \texttt{"welt"}
			\item Konkatenation von Zeichenketten: \\[6pt] 
			\begin{codebox}
"hallo " ++ "welt" = "hallo welt"
			\end{codebox}
		\end{itemize}
	\end{frame}

\begin{frame}[t, fragile] \frametitle{Aufgabe 2 -- Teil (a)}
	\textbf{Präfix -- Test}
	
	\texttt{isPrefix :: String -> String -> Bool}
	
	\pause \bigskip
	
	\begin{codebox}
isPrefix :: String -> String -> Bool
isPrefix [] _ = True
isPrefix _ [] = False
isPrefix (p:ps) (c:cs) = p == c && isPrefix ps cs
	\end{codebox}
\end{frame}

\begin{frame}[t, fragile] \frametitle{Aufgabe 2 -- Teil (b)}
	\textbf{Vorkommen eines Patterns zählen}
	
	\texttt{countPattern :: String -> String -> Int}
	
	\pause\bigskip

	\begin{codebox}
countPattern :: String -> String -> Int
countPattern "" "" = 1
countPattern _  "" = 0
countPattern xs yys@(y:ys)
	| isPrefix xs yys = 1 + countPattern xs ys
	| otherwise       = countPattern xs ys
	\end{codebox}
\end{frame}

%%%%%%%%%%%%%%%%%%%%%%%%%%%%%%%%%%%%%%%%%%%%%%%%%%%%%%%%%%%%%%%%%%%%%%%%%%%%%%%%%%%%

\section{Aufgabe 3 \\ \textit{\normalsize Algebraische Datentypen}}

\begin{frame}[fragile] \frametitle{Algebraische Datentypen}
	\footnotesize
	\begin{itemize}
		\item Ziel: problemspezifische Datenkonstruktoren
		\item z.B. in \textit{C}: Aufzählungstypen
		\item funktionale Programmierung: algebraische Datentypen
	\end{itemize}

	\textbf{Aufbau:}
	\begin{codebox}
data Typename 
	= Con1 t11 ... t1k1
	| Con2 t21 ... t2k2
	| ...
	| Conr tr1 ... trkr
	\end{codebox}
	\begin{itemize}
		\item \texttt{Typename} ist ein Name (Großbuchstabe)
		\item \texttt{Con1, ... Conr} sind Datenkonstruktoren (Großbuchstabe)
		\item \texttt{tij} sind Typnamen (Großbuchstaben)
	\end{itemize}
\end{frame}

\begin{frame}[fragile] \frametitle{Algebraische Datentypen -- Beispiele}
	
	\begin{codebox}
data Typename 
	= Con1 t11 ... t1k1
	| Con2 t21 ... t2k2
	| ...
	| Conr tr1 ... trkr
	\end{codebox}
	
	\begin{codebox}
data Season = Spring | Summer | Autumn | Winter
	\end{codebox}
	\pause
	\begin{codebox}
goSkiing :: Season -> Bool
goSkiing Winter = True
goSkiing _      = False
	\end{codebox}
	\pause
	\begin{codebox}
data Weather = Sunny Int Int Bool | Cloudy Float 
                                  | Rainy String Int 
	\end{codebox}
\end{frame}

\begin{frame}[t, fragile] \frametitle{Aufgabe 3 -- Teil (a)}
	\begin{codebox}
data BinTree = Branch Int BinTree BinTree | Nil
	\end{codebox}
	\pause
	
	Ein Beispielbaum: \\[6pt]
	\begin{codebox}
mytree :: BinTree 
mytree = Branch 0 
         ( Nil )
         ( Branch 3 
           ( Branch 1 Nil Nil )
           ( Branch 5 Nil Nil )
         )
	\end{codebox}
	... erfüllt die Suchbaumeigenschaft.
\end{frame}

\begin{frame}[t, fragile] \frametitle{Aufgabe 3 -- Teil (b)}
	\textbf{Test auf Baum-Gleichheit}
	
	\pause
	
	\begin{codebox}
data BinTree = Branch Int BinTree BinTree | Nil
equal :: BinTree -> BinTree -> Bool
	\end{codebox}
	
	\bigskip \pause
	
	\begin{codebox}
equal :: BinTree -> BinTree -> Bool
equal Nil              Nil              = True
equal Nil              (Branch y l2 r2) = False
equal (Branch x l1 r1) Nil              = False
equal (Branch x l1 r1) (Branch y l2 r2)
	= (x == y) && (equal l1 l2) && (equal r1 r2)
	\end{codebox}
\end{frame}

\begin{frame}[t, fragile] \frametitle{Aufgabe 3 -- Teil (c)}
	\textbf{Einfügen von Schlüsseln in einen Binärbaum}
	
	\begin{codebox}
data BinTree = Branch Int BinTree BinTree | Nil 
insert :: BinTree -> [Int] -> BinTree
	\end{codebox}
	
	\bigskip \pause
	
	\begin{codebox}
insert :: BinTree -> [Int] -> BinTree
insert t     [] = t
insert t (x:xs) = insert t' xs
    where t' = insertSingle t x
        insertSingle Nil            x = Branch x Nil Nil
        insertSingle (Branch y l r) x
            | x < y     = Branch y (insertSingle l x) r
            | otherwise = Branch y l (insertSingle r x)
	\end{codebox}
\end{frame}



\begin{frame} \frametitle{Ende}
	\centering
	\textbf{Fragen?}
\end{frame}

\end{document}