\documentclass{beamer}
\usepackage{../tut-slides}
\usepackage{../mathoperatorsAuD}

\usepackage{csquotes}
\usepackage{cancel}

\usepackage{amsmath,amssymb}
\usepackage{enumerate}
\usepackage[inline]{enumitem} 		%customize label
\newcommand{\labelitemi}{\raisebox{1pt}{\scalebox{.9}{$\blacktriangleright$}}}
\newcommand{\labelitemii}{$\vartriangleright$}
\newcommand{\labelitemiii}{--}

\usepackage{booktabs}
\usepackage{tabularx}
\usepackage{tabu}
\newcommand*\head{\rowfont{\bfseries}}
\newcommand*{\tw}{\rowfont{\ttfamily}}

\renewcommand{\tabularxcolumn}[1]{>{\hspace{0pt}}m{#1}}

\usepackage{listings}
\usepackage[skins]{tcolorbox}
\tcbuselibrary{listings}
\newtcblisting{codebox}{%
	top=0pt,
	bottom=0pt,
	left=0pt,
	right=0pt,
	boxsep=5pt,
	enhanced,clip upper,
	colback=white,
	colframe=cddarkblue,
	fonttitle=\bfseries,
	listing only}
\newtcblisting{smallcodebox}{%
	top=0pt,
	bottom=0pt,
	left=0pt,
	right=0pt,
	boxsep=5pt,
	enhanced,clip upper,
	colback=white,
	colframe=cddarkblue,
	fonttitle=\bfseries,
	fontupper=\tiny,
	listing only}

\lstdefinestyle{bg}{ 
	basicstyle=\footnotesize\ttfamily,        % the size of the fonts that are used for the code
	breakatwhitespace=false,         % sets if automatic breaks should only happen at whitespace
	breaklines=true,                 % sets automatic line breaking
	commentstyle=\itshape,    	     % comment style
	escapeinside={\%*}{\%*},          % if you want to add LaTeX within your code
	extendedchars=true,              % lets you use non-ASCII characters; for 8-bits encodings only, does not work with UTF-8
	firstnumber=1,                % start line enumeration with line 1000
	frame=L,
	backgroundcolor=\color{cdgray!10},
	%	keywordstyle=\bfseries,       % keyword style
	language=Haskell,                 % the language of the code
	numbers=left,                    % where to put the line-numbers; possible: (none, left, right)
	numbersep=8pt,                   % how far the line-numbers are from the code
	numberstyle=\tiny\color{cdgray!50}, % the style that is used for the line-numbers
	rulecolor=\color{cddarkblue}, 
	showstringspaces=false,
	tabsize=2,	                   % sets default tabsize to 2 spaces
}


\begin{document}	
	\title{Programmierung}
	\subtitle{Übung 4: Funktionen höherer Ordnung}
	\author{Eric Kunze}
	\email{eric.kunze@tu-dresden.de}
%	\city{TU Dresden}
	\date{02. Mai 2022}
%	\institute{Lehrstuhl für Grundlagen der Programmierung}
%	\titlegraphic{\includegraphics[width=2cm]{../TUD-white.pdf}}
	
	\maketitle
	

%%%%%%%%%%%%%%%%%%%%%%%%%%%%%%%%%%%%%%%%%%%%%%%%%%%%%%%%%%%%%%%%%%%%%%%%%%%%%

\section{Übungsblatt 4 --- Aufgabe 1 \\ \textit{\normalsize Algebraische Datentypen}}

\begin{frame}[t, fragile] \frametitle{Aufgabe 1 -- Teil (a)}
	\textbf{Anzahl der Blätter}
	
	\begin{lstlisting}[style=bg]
data RoseTree = Node Int [RoseTree]
countLeaves :: RoseTree -> Int
	\end{lstlisting}
	
	
	\pause \bigskip
	
	\begin{lstlisting}[style=bg]
countLeaves :: RoseTree -> Int
countLeaves (Node _ [] )    = 1
countLeaves (Node _ [t])    = countLeaves t
countLeaves (Node x (t:ts)) 
	= countLeaves t + countLeaves (Node x ts)
	\end{lstlisting}
\end{frame}

\begin{frame}[t, fragile] \frametitle{Aufgabe 1 -- Teil (b)}
	\textbf{gerade Anzahl an Kindern}
	
	\begin{lstlisting}[style=bg]
data RoseTree = Node Int [RoseTree]
evenNodes :: RoseTree -> Bool
	\end{lstlisting}
	
	
	\pause \bigskip
	\begin{onlyenv}<2>
		\begin{lstlisting}[style=bg]
evenNodes :: RoseTree -> Bool
evenNodes (Node _  []       ) = True
evenNodes (Node x  [t]      ) = False
evenNodes (Node x (t1:t2:ts)) 
	= evenNodes (Node x ts) && evenNodes t1 && 
	  evenNodes t2
		\end{lstlisting}
	\end{onlyenv}

	\begin{onlyenv}<3>
		\begin{lstlisting}[style=bg]
evenNodes' :: RoseTree -> Bool
evenNodes' (Node _ []) = True
evenNodes' (Node _ ts)
	= mod (length ts) 2 == 0 && evenNodes'' ts
	where
		evenNodes'' :: [RoseTree] -> Bool
		evenNodes'' []     = True
		evenNodes'' (t:ts) 
			= evenNodes' t && evenNodes'' ts
		\end{lstlisting}
	\end{onlyenv}
\end{frame}


%%%%%%%%%%%%%%%%%%%%%%%%%%%%%%%%%%%%%%%%%%%%%%%%%%%%%%%%%%%%%%%%%%%%%%%%%%%%%%%%%%%%

\section{Übungsblatt 4 --- Aufgaben 2 \& 3 \\ \textit{\normalsize Funktionen höherer Ordnung}}


\begin{frame}[fragile] \frametitle{Funktionen}
	\footnotesize
	Wir kennen bereits einige Möglichkeiten Funktionen zu notieren. Hier seien einige weitere erwähnt.
	\begin{description}
		\item[anonyme Funktionen.] Funktionen ohne konkreten Namen \\
		z.B. \lstinline[basicstyle=\ttfamily\normalsize]|(\x -> x+1)| ist die Addition mit \texttt{1} 
		\begin{lstlisting}[style=bg]
(\x -> x+1) 4 = 5
		\end{lstlisting}
		%
		\pause
		%
		\item[Operator $\leftrightarrow$ Funktion] Aus Operatoren (wie z.B. \texttt{+}) kann man eine Funktion machen und vice versa.
		\begin{itemize}
			\item Operator $\to$ Funktion: Klammern
			\begin{lstlisting}[style=bg]
(+) :: Int -> Int -> Int
(+) x y =  x + y
			\end{lstlisting}
			\item Funktion $\to$ Operator: Backticks \texttt{`...`}
			\begin{lstlisting}[style=bg]
5 `mod` 2 = 1
			\end{lstlisting}
		\end{itemize}
	\end{description}
\end{frame}

\begin{frame}[fragile] \frametitle{Funktionskomposition}
	Analog zur mathematischen Notation $f = g \circ h$ für $f(x) = g(h(x))$ versteht auch Haskell das Kompositionsprinzip mit dem Operator \texttt{.} \\
	z.B. 
	\begin{lstlisting}[style=bg]
sqAdd :: Int -> Int
sqAdd = (^2) . (+ 5)
	\end{lstlisting} 
	statt \lstinline[basicstyle=\ttfamily\normalsize]|sqAdd x = (x + 5)^2| für das Quadrat des fünften Nachfolgers
\end{frame}

\begin{frame}[fragile] \frametitle{Partielle Applikation}
	\footnotesize
	Funktionen müssen nicht immer mit allen Argumenten versorgt werden. Lässt man (hintere) Argumente weg, so spricht man von Unterversorgung.
	Die Modulo Funktion hat eigentlich zwei Argumente. Lassen wir das zweite Argument weg, so liefert dies uns eine neue Funktion, die noch ein Argument entgegennimmt und sodann die Restberechnung ausführt.
	\begin{lstlisting}[style=bg]
mod :: Int -> Int -> Int
mod m n = ...

mod 10 :: Int -> Int
(mod 10) n = mod 10 n
	\end{lstlisting}
	\pause
	\begin{lstlisting}[style=bg]
(> 3) :: Int -> Bool
(> 3) x = x > 3
	\end{lstlisting}
\end{frame}

\begin{frame}[fragile] \frametitle{Funktionen höherer Ordnung --- map}
	\footnotesize
	Funktionen können als Argumente von Funktionen auftreten. Wir lernen drei Basics kennen: \pause
	
	\textbf{Die Funktion map}
	\begin{itemize}
		\item \texttt{map} ermöglicht es eine Funktion \texttt{f} auf alle Elemente einer Liste anzuwenden \\[6pt]
		\begin{lstlisting}[style=bg]
map :: (Int -> Int) -> [Int] -> [Int] 
map f [] = [] 
map f (x:xs) = f x : map f xs
		\end{lstlisting}
		\item \textit{Beispiel.} \\[6pt]
		\begin{lstlisting}[style=bg]
map square [1,2,7,12,3,20] = [1,4,49,144,9,400]
		\end{lstlisting}
	\end{itemize}
\end{frame}

\begin{frame}[fragile] \frametitle{Funktionen höherer Ordnung --- filter}
	\footnotesize
	\textbf{Die Funktion filter}
	\begin{itemize}
		\item \texttt{filter p xs} liefert eine Liste, die genau die Elemente von \texttt{xs} enthält, welche das Prädikat \texttt{p} erfüllen \\[6pt]
		\begin{lstlisting}[style=bg]
filter :: (a -> Bool) -> [a] -> [a] 
filter p xs = [ x | x <- xs, p x]
		\end{lstlisting}
		\item \textit{Beispiel.} \\[6pt]
		\begin{lstlisting}[style=bg]
filter odd [1,2,7,12,3,20] = [1,7,3]
		\end{lstlisting}
	\end{itemize}
\end{frame}

\begin{frame}[fragile] \frametitle{Funktionen höherer Ordnung --- foldr}
	\footnotesize
	\textbf{Die Funktion foldr}
	\begin{itemize}
		\item \texttt{foldr f z xs} faltet eine Liste \texttt{xs} und verknüpft jeweils durch die Funktion \texttt{f}; gestartet wird mit \texttt{z} und dem rechtesten Element \\[6pt]
		\begin{lstlisting}[style=bg]
foldr :: (a -> b -> b) -> b -> [a] -> b
foldr f z []     = z 
foldr f z (x:xs) = f x (foldr f z xs) 
		\end{lstlisting}
		\item \emph{Beispiel.} \\[6pt]
		\begin{lstlisting}[style=bg]
foldr (+) 3 [1,2,3,4,5] = 18
length xs = foldr (+) 0 (map (\x -> 1) xs)
		\end{lstlisting}
	\end{itemize}
\end{frame}


\begin{frame}[fragile] \frametitle{Funktionen höherer Ordnung -- Übersicht}
	\footnotesize
	\begin{itemize}
		\item \texttt{map} wendet Funktion auf alle Listenelemente an \\[6pt]
		\begin{lstlisting}[style=bg]
map :: (a -> b) -> [a] -> [b]
map f [] = []
map f (x:xs) = f x : map f xs
		\end{lstlisting}
		\item \texttt{filter} wählt Listenelemente anhand einer Funktion aus \\[6pt]
		\begin{lstlisting}[style=bg]
filter :: (a -> Bool) -> [a] -> [a]
filter p xs = [ x | x <- xs, p x]
		\end{lstlisting}
		\item \texttt{foldr} faltet eine Liste mit Verknüpfungsfunktion (von rechts beginnend) \\[6pt]
		\begin{lstlisting}[style=bg]
foldr :: (a -> b -> b) -> b -> [a] -> b
foldr f z []     = z
foldr f z (x:xs) = f x (foldr f z xs) 
		\end{lstlisting}
	\end{itemize}
\end{frame}

\begin{frame}[t, fragile] \frametitle{Aufgabe 2}
	\footnotesize
	\textbf{Produkt der Quadrate aller geraden Zahlen einer Liste} \\[6pt]
	\begin{lstlisting}[style=bg]
f :: [Int] -> Int
	\end{lstlisting}
	
	\pause \bigskip 
	
	\begin{lstlisting}[style=bg]
f xs 
  = foldr (+) 0 (map (^2) (filter (`mod` 2) == 0) xs))
	\end{lstlisting}
	\pause
	\begin{lstlisting}[style=bg]
f' xs = foldr (*) 1 (map (^2) (filter even xs))
	\end{lstlisting}
	\pause
	\begin{lstlisting}[style=bg]
f'' = foldr (*) 1 . map (^2) . filter even
	\end{lstlisting}
	\begin{lstlisting}[style=bg]
f''' 
  = foldr (*) 1 . map (^2) . filter ((== 0) . (`mod` 2))
	\end{lstlisting}
\end{frame}


%%%%%%%%%%%%%%%%%%%%%%%%%%%%%%%%%%%%%%%%%%%%%%%%%%%%%%%%%%%%%%%%%%%%%%%%%%%%%%%%%%%


\begin{frame}[t, fragile] \frametitle{Aufgabe 3}
	\textbf{Faltung einer Liste von \textit{links}} \\[6pt]
	\begin{lstlisting}[style=bg]
foldleft ::  (Int -> Int -> Int) -> Int -> [Int] -> Int
	\end{lstlisting}
	
	
	\pause \bigskip
	
	\begin{lstlisting}[style=bg]
foldleft :: (Int -> Int -> Int) -> Int -> [Int] -> Int
foldleft f x []     = x
foldleft f x (y:ys) = foldleft f (f x y) ys
	\end{lstlisting}
\end{frame}


\begin{frame} \frametitle{Ende}
	\centering
	\textbf{Fragen?}
\end{frame}

\end{document}

