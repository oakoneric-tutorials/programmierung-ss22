\documentclass{beamer}
\usepackage{../tut-slides}
\usepackage{../mathoperatorsAuD}

\usepackage{csquotes}
\usepackage{cancel}

\usepackage{amsmath,amssymb}


\usepackage{booktabs}
\usepackage{tabularx}
\usepackage{tabu}
\newcommand*\head{\rowfont{\bfseries}}
\newcommand*{\tw}{\rowfont{\ttfamily}}

\renewcommand{\tabularxcolumn}[1]{>{\hspace{0pt}}m{#1}}

\usepackage{listings}
\usepackage[skins]{tcolorbox}
\tcbuselibrary{listings}

\newtcblisting{codebox}{%
	top=-6pt,
	bottom=-6pt,
	left=0pt,
	right=0pt,
	boxsep=5pt,
	enhanced,clip upper,
	colback=white,
	colframe=cddarkblue,
	fonttitle=\bfseries,
	listing only,
	listing options={basicstyle=\scriptsize\ttfamily}
}
\newtcblisting{smallcodebox}{%
	top=0pt,
	bottom=0pt,
	left=0pt,
	right=0pt,
	boxsep=5pt,
	enhanced,clip upper,
	colback=white,
	colframe=cddarkblue,
	listing only}


\newcommand{\col}[1]{\textcolor{cdpurple}{#1}}
\newcommand{\step}[2][]{\ensuremath{\overset{{#1} (\text{#2})}{=}}}
\newcommand*{\astep}[2][]{\ensuremath{\overset{{#1} (\text{#2})}&{=}}}

\begin{document}	
	\title{Programmierung}
	\subtitle{Übung 5: Induktion}
	\author{Eric Kunze}
	\email{eric.kunze@tu-dresden.de}
%	\city{TU Dresden}
	\date{}
%	\institute{Lehrstuhl für Grundlagen der Programmierung}
%	\titlegraphic{\includegraphics[width=2cm]{../TUD-white.pdf}}
	
	\maketitle
	

%%%%%%%%%%%%%%%%%%%%%%%%%%%%%%%%%%%%%%%%%%%%%%%%%%%%%%%%%%%%%%%%%%%%%%%%%%%%%

\begin{frame}[fragile] \frametitle{Inhalt}
	\begin{enumerate}
		\item Funktionale Programmierung
		\begin{enumerate}
			\item Einführung in Haskell
			\item Listen \& Algebraische Datentypen
			\item Funktionen höherer Ordnung
			\item Typpolymorphie \& Unifikation
			\item \textbf{Beweis von Programmeigenschaften}
			\item $\lambda$ -- Kalkül
		\end{enumerate}
		\item Logikprogrammierung
		\item Implementierung einer imperativen Programmiersprache
		\item Verifikation von Programmeigenschaften
		\item H${}_\text{0}$ -- ein einfacher Kern von Haskell
	\end{enumerate}
\end{frame}



\section{Induktionsbeweise \\ \textit{\normalsize Aufgaben 1 und 2}}

\begin{frame} \frametitle{Vollständige Induktion auf $\N$}
	\footnotesize
	
	\textbf{Definition:} natürliche Zahlen $\N \defeq \menge{0, 1, \dots}$
	\begin{equation*}
		\begin{aligned}
			\text{Basisfall: } \quad &&0 &\in \N \\
			\text{Rekursionsfall: }\quad &&x + 1 &\in \N
			\quad \text{ für } \quad x \in \N
		\end{aligned}
	\end{equation*}
	
	\textbf{Beweis von Eigenschaften:}
	Eigenschaft $=$ Prädikat $P$
	\begin{center}
		\fbox{zu zeigen: \hspace{2em} für alle $x \in \N$ gilt $P(x)$}
	\end{center}
	
	\textbf{vollständige Induktion:}
	\begin{itemize}
		\item \textcolor{cdblue}{Induktionsanfang}: \\
		zeige $P(x)$ für $x = 0$
		\item \textcolor{cdblue}{Induktionsvoraussetzung}:\\
		Sei $x \in \N$, sodass $P(x)$ gilt.
		\hfill
		\textcolor{cdgray}{$P(x)$ gilt noch nicht für \textit{alle} $x \in \N$}
		\item \textcolor{cdblue}{Induktionsschritt}: \\
		zeige $P(x+1)$ unter Nutzung der Induktionsvoraussetzung
	\end{itemize}
\end{frame}

\begin{frame} \frametitle{Induktion auf Listen}
\footnotesize

	\textbf{Erinnerung:} Rekursion über Listen \texttt{xs}
	\begin{equation*}
		\begin{aligned}
			\text{Basisfall: } \quad \texttt{xs} &= \texttt{[]} \\
			\text{Rekursionsfall: }\quad \texttt{xs} &= \texttt{(y:ys)} \quad \text{ für } \quad \texttt{ys :: [a]}
		\end{aligned}
	\end{equation*}
	\pause
	
	\textbf{Beweis von Programmeigenschaften:}
	Eigenschaft $=$ Prädikat $P$
	\begin{center}
		\fbox{zu zeigen: \hspace{2em} für alle \texttt{xs :: [a]} gilt $P(\texttt{xs})$}
	\end{center}
	\pause
	
	\textbf{Induktion auf Listen:} \vspace{-0.5\baselineskip}
	\begin{itemize}
		\item \textcolor{cdblue}{Induktionsanfang}: \\
		zeige $P(\texttt{xs})$ für \texttt{xs == []}
		\pause
		\item \textcolor{cdblue}{Induktionsvoraussetzung}:\\
		Sei \texttt{xs :: [a]} eine Liste für die $P(\texttt{xs})$ gilt. \pause
		\item \textcolor{cdblue}{Induktionsschritt}: \\
		zeige $P(\texttt{x:xs})$ für alle \texttt{x :: a} unter Nutzung der Induktionsvoraussetzung
		\pause
	\end{itemize}
	
	\fbox{\emph{Allgemeiner Hinweis:} Es müssen immer \alert{alle} Variablen quantifiziert werden!}
\end{frame}

\begin{frame} \frametitle{Strukturelle Induktion}
	\footnotesize
	
	\textbf{Erinnerung:} Rekursion über Bäume
	\begin{equation*}
		\begin{aligned}
			\text{Basisfall: } \quad &\texttt{Nil} \text{ oder } \texttt{Leaf x} \text{ für } \texttt{x :: a} \\
			\text{Rekursionsfall: }\quad &\texttt{Branch x l r} \text{ für } \texttt{x :: a} \text{ und } \texttt{l,r :: BinTree a}
		\end{aligned}
	\end{equation*}
	\pause

	\begin{center}
		\fbox{zu zeigen: \hspace{2em} für alle \texttt{t :: BinTree a} gilt $P(\texttt{t})$}
	\end{center}
	\pause
	
	\textbf{strukturelle Induktion:} \vspace{-0.5\baselineskip}
	\begin{itemize}
		\item \textcolor{cdblue}{Induktionsanfang}: \\
		zeige $P(\texttt{t})$ für \texttt{t == Nil} oder \texttt{t == Leaf x} für alle \texttt{x :: a}
		\pause
		\item \textcolor{cdblue}{Induktionsvoraussetzung}:\\
		Seien \texttt{l, r :: BinTree a} zwei Bäume, sodass $P(\texttt{l})$ und $P(\texttt{r})$ gilt. \pause
		\item \textcolor{cdblue}{Induktionsschritt}: \\
		zeige $P(\texttt{Branch x l r})$ für alle \texttt{x :: a} unter Nutzung der Induktionsvoraussetzung
		\pause
	\end{itemize}
	
	\fbox{\emph{Allgemeiner Hinweis:} Es müssen immer \alert{alle} Variablen quantifiziert werden!}
\end{frame}

\begin{frame} \frametitle{Fehlerquellen}
	\footnotesize
	\begin{itemize}
		\item kein Induktionsprinzip
		\item IV wird im Induktionsschritt nicht verwendet
		\item fehlende Quantifizierung (nur Gleichungen bringen kaum Punkte)
		\item \textit{Missachtung freier Variablen} \pause
		\item zu beweisende Eigenschaft $P$ wird für \texttt{xs} angenommen, um sie dann im Induktionsschritt nochmal für \texttt{xs} zu beweisen –- eine Tautologie
		\item Annahme, dass $P$ bereits für alle Listen gilt, um es dann für \texttt{x:xs} nochmal zu zeigen		
	\end{itemize}
\end{frame}

\begin{frame} \frametitle{Aufgabe 1}
	\scriptsize
	Zu zeigen ist die Gleichung
	\begin{equation*}
		\texttt{sum (foo xs) = 2 * sum xs - length xs} \qquad \text{für alle } \texttt{xs :: Int}
	\end{equation*}
	mittels Induktion über Listen.
	
	\textbf{Induktionsanfang:} Sei \texttt{xs == []}. 
	\begin{align*}
		\intertext{linke Seite: }
		\texttt{sum (foo [])} &\step{2} \texttt{sum [] \step{6} 0} \\
		\intertext{rechte Seite: }
		\texttt{2 * sum [] - length []} &\step{10} \texttt{2 * sum [] - 0 \step{6} 2 * 0 - 0 = 0}
	\end{align*}
	
	\textbf{Induktionsvoraussetzung:} Sei \texttt{xs :: [Int]}, sodass
	\begin{equation*}
		\texttt{sum (foo xs) = 2 * sum xs - length xs}
	\end{equation*}
	gilt.
\end{frame}

\begin{frame} \frametitle{Aufgabe 1 (Fortsetzung)}
	\scriptsize
	
	\textbf{Induktionsschritt:} Sei \texttt{x :: Int}. Es gilt
	\begin{align*}
		\texttt{sum (foo (x:xs))} \astep{3} \texttt{sum (x : x : (-1) : foo xs)} \\
		\astep[3*]{7} \texttt{x + x + (-1) + sum (foo xs)} \\
		\astep{IV} \texttt{x + x + (-1) + 2 * sum xs - length xs} \\
		\astep{Komm.} \texttt{2 * x + 2 * sum xs - 1 - length xs)} \\
		\astep{Dist.} \texttt{2 * (x + sum xs) - (1 + length xs)} \\
		\astep{7} \texttt{2 * sum (x:xs) - (1 + length xs)} \\
		\astep{11} \texttt{2 * sum (x:xs) - length (x:xs)} 
	\end{align*}
\end{frame}

\begin{frame} \frametitle{Aufgabe 2 --- Teil (a)}
	\scriptsize
	Sei \texttt{a} ein beliebiger Typ. Zu zeigen ist die Gleichung
	\begin{equation}
		\begin{aligned}
			\texttt{[x] ++ rev ys ++ rev xs = rev (xs ++ ys ++ [x])} \\ \text{ für alle } \texttt{xs, ys :: [a]}
		\end{aligned}
		\tag{H3}
	\end{equation}
	
	Der Beweis funktioniert ohne Induktion:
	\begin{align*}
		\texttt{[x] ++ rev ys ++ rev xs}
		\astep{Ass.} 
		\texttt{[x] ++ (rev ys ++ rev xs)} \\
		\astep{H2} \texttt{[x] ++ rev (xs ++ ys)} \\
		\astep{H1} \texttt{rev [x] ++ rev (xs ++ ys)} \\
		\astep{H2} \texttt{rev ((xs ++ ys) ++ [x])} \\
		\astep{Ass.} \texttt{rev (xs ++ ys ++ [x])}
	\end{align*}
\end{frame}

\begin{frame} \frametitle{Aufgabe 2 --- Teil (b)}
	\scriptsize
	Sei \texttt{a} ein beliebiger Typ. Zu zeigen ist die Gleichung
	\begin{equation*}
		\texttt{preOrder t = rev (mPostOrder t)} \qquad \text{für alle} \qquad \texttt{t :: BinTree a}
	\end{equation*}
	mittels struktureller Induktion.
	
	\textbf{Induktionsanfang:} Sei \texttt{x :: a} beliebig und \texttt{t = Leaf x}. 
	\begin{align*}
		\text{linke Seite: } \quad &\texttt{preOrder (Leaf x)} \step{4} \texttt{[x]} \\
		\text{rechte Seite: } \quad & \texttt{rev (mPostOrder (Leaf x))} \step{8} \texttt{rev [x] \step{H1} [x]}
	\end{align*}
	
	\textbf{Induktionsvoraussetzung:} Seien \texttt{l, r:: BinTree a}, sodass
	\begin{align}
		\texttt{preOrder l} &= \texttt{rev(mPostOrder l)} \tag{IV1} \\
		\texttt{preOrder r} &= \texttt{rev(mPostOrder r)} \tag{IV2}
	\end{align}
	gilt.
\end{frame}

\begin{frame} \frametitle{Aufgabe 2 -- Teil (b) (Fortsetzung)}
	\scriptsize
	
	\textbf{Induktionsschritt:} Sei \texttt{x :: a} beliebig. Es gilt
	\begin{align*}
		\texttt{preOrder (Node x l r)} \astep{5} \texttt{[x] ++ preOrder l ++ preOrder r} \\
		\astep{IV1} \texttt{[x] ++ rev(mPostOrder l) ++ preOrder r} \\
		\astep{IV2} \texttt{[x] ++ rev(mPostOrder l) ++ rev(mPostOrder r)} \\
		\astep{H3} \texttt{rev(mPostOrder r ++ mPostOrder l ++ [x])} \\
		\astep{9} \texttt{rev(mPostOrder (Node x l r))}
	\end{align*}
\end{frame}

\begin{frame} \frametitle{Ende}
	\centering
	\textbf{Fragen?}
\end{frame}

\end{document}

