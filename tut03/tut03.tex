\documentclass{beamer}
\usepackage{../tut-slides}
\usepackage{../mathoperatorsAuD}

\usepackage{csquotes}
\usepackage{cancel}

\usepackage{amsmath,amssymb}
\usepackage{enumerate}
\usepackage[inline]{enumitem} 		%customize label
\newcommand{\labelitemi}{\raisebox{1pt}{\scalebox{.9}{$\blacktriangleright$}}}
\newcommand{\labelitemii}{$\vartriangleright$}
\newcommand{\labelitemiii}{--}

\usepackage{booktabs}
\usepackage{tabularx}
\usepackage{tabu}
\newcommand*\head{\rowfont{\bfseries}}
\newcommand*{\tw}{\rowfont{\ttfamily}}

\renewcommand{\tabularxcolumn}[1]{>{\hspace{0pt}}m{#1}}

\usepackage[skins]{tcolorbox}
\tcbuselibrary{listings}
\newtcblisting{codebox}{%
	top=0pt,
	bottom=0pt,
	left=0pt,
	right=0pt,
	boxsep=5pt,
	enhanced,clip upper,
	colback=white,
	colframe=cddarkblue,
	fonttitle=\bfseries,
	listing only}
\newtcblisting{smallcodebox}{%
	top=0pt,
	bottom=0pt,
	left=0pt,
	right=0pt,
	boxsep=5pt,
	enhanced,clip upper,
	colback=white,
	colframe=cddarkblue,
	fonttitle=\bfseries,
	fontupper=\tiny,
	listing only}

\lstdefinestyle{bg}{ 
	basicstyle=\footnotesize\ttfamily,        % the size of the fonts that are used for the code
	breakatwhitespace=false,         % sets if automatic breaks should only happen at whitespace
	breaklines=true,                 % sets automatic line breaking
	commentstyle=\itshape,    	     % comment style
	escapeinside={\%*}{\%*},          % if you want to add LaTeX within your code
	extendedchars=true,              % lets you use non-ASCII characters; for 8-bits encodings only, does not work with UTF-8
	firstnumber=1,                % start line enumeration with line 1000
	frame=L,
	backgroundcolor=\color{cdgray!10},
	%	keywordstyle=\bfseries,       % keyword style
	language=Haskell,                 % the language of the code
	numbers=left,                    % where to put the line-numbers; possible: (none, left, right)
	numbersep=8pt,                   % how far the line-numbers are from the code
	numberstyle=\tiny\color{cdgray!50}, % the style that is used for the line-numbers
	rulecolor=\color{cddarkblue}, 
	showstringspaces=false,
	tabsize=2,	                   % sets default tabsize to 2 spaces
}

\begin{document}	
	\title{Programmierung}
	\subtitle{Übung 3: Zeichenketten \& Bäume}
	\author{Eric Kunze}
	\email{eric.kunze@tu-dresden.de}
	\city{TU Dresden}
	\date{25.04.2022}
%	\institute{Lehrstuhl für Grundlagen der Programmierung}
%	\titlegraphic{\includegraphics[width=2cm]{../TUD-white.pdf}}
	
	\maketitle


%%%%%%%%%%%%%%%%%%%%%%%%%%%%%%%%%%%%%%%%%%%%%%%%%%%%%%%%%%%%%%%%%%%%%%%%%%%%%%%%%%%%

\section{Aufgabe 1 \\ \textit{\normalsize Zeichen \& Zeichenketten}}

	\begin{frame}[fragile] \frametitle{Zeichen \& Zeichenketten}
		\textbf{Zeichen}
		\begin{itemize}
			\item Datentyp \texttt{Char}
			\item Eingabe in einfachen Anführungszeichen
			\item z.B. \texttt{'a'}, \texttt{'e'}, \texttt{'3'}
		\end{itemize}
		\pause
		\textbf{Zeichenketten}
		\begin{itemize}
			\item Datentyp \texttt{String = [Char]}
			\item Eingabe in doppelten Anführungszeichen
			\item z.B. \texttt{"hallo"}, \texttt{"welt"}
			\item Konkatenation von Zeichenketten: \\[6pt] 
			\begin{lstlisting}[style=bg]
"hallo " ++ "welt" = "hallo welt"
			\end{lstlisting}
		\end{itemize}
	\end{frame}

\begin{frame}[t, fragile] \frametitle{Aufgabe 1 -- Teil (a)}
	\textbf{Präfix -- Test}
	
	\texttt{isPrefix :: String -> String -> Bool}
	
	\pause \bigskip
	
	\begin{lstlisting}[style=bg]
isPrefix :: String -> String -> Bool
isPrefix [] _ = True
isPrefix _ [] = False
isPrefix (p:ps) (c:cs) = p == c && isPrefix ps cs
	\end{lstlisting}
\end{frame}

\begin{frame}[t, fragile] \frametitle{Aufgabe 1 -- Teil (b)}
	\textbf{Vorkommen eines Patterns zählen}
	
	\texttt{countPattern :: String -> String -> Int}
	
	\pause\bigskip

	\begin{lstlisting}[style=bg]
countPattern :: String -> String -> Int
countPattern "" "" = 1
countPattern _  "" = 0
countPattern xs yys@(y:ys)
	| isPrefix xs yys = 1 + countPattern xs ys
	| otherwise       = countPattern xs ys
	\end{lstlisting}
\end{frame}

%%%%%%%%%%%%%%%%%%%%%%%%%%%%%%%%%%%%%%%%%%%%%%%%%%%%%%%%%%%%%%%%%%%%%%%%%%%%%%%%%%%%

\section{Aufgabe 2 \\ \textit{\normalsize Algebraische Datentypen}}

\begin{frame}[fragile] \frametitle{Algebraische Datentypen}
	\footnotesize
	\begin{itemize}
		\item Ziel: problemspezifische Datenkonstruktoren
		\item z.B. in \textit{C}: Aufzählungstypen
		\item funktionale Programmierung: algebraische Datentypen
	\end{itemize}

	\textbf{Aufbau:}
	\begin{lstlisting}[style=bg]
data Typename 
	= Con1 t11 ... t1k1
	| Con2 t21 ... t2k2
	| ...
	| Conr tr1 ... trkr
	\end{lstlisting}
	\begin{itemize}
		\item \texttt{Typename} ist ein Typkonstruktor
		\item \texttt{Con1, ... Conr} sind Datenkonstruktoren
		\item \texttt{tij} sind Typnamen
	\end{itemize}
	Konstruktoren beginnen mit Großbuchstaben.
\end{frame}

\begin{frame}[fragile] \frametitle{Algebraische Datentypen -- Beispiele}
	
	\begin{lstlisting}[style=bg]
data Season = Spring | Summer | Autumn | Winter %* \pause %*

goSkiing :: Season -> Bool
goSkiing Winter = True
goSkiing _      = False %* \pause %*

data Weather = Sunny Int Int Bool | Cloudy Float 
                                  | Rainy String Int %* \pause %*
                           
data Bool =  True | False %* \pause %*

data BinTree = Branch Int BinTree BinTree | Nil
	\end{lstlisting}
	\begin{itemize}
		\item Konstruktor \texttt{Branch}: erzeugt Knoten mit Beschriftung und zwei Kindern  
		\hfill $\leadsto$ \textit{Rekursionsfall}
		\item Konstruktor \texttt{Nil}: erzeugt leeren Baum \hfill $\leadsto$ \textit{Basisfall}
	\end{itemize}
\end{frame}

\begin{frame}[fragile] \frametitle{Algebraische Datentypen -- Rekursion}
	\small
	\begin{itemize}
		\item Basisfall: nulläre Wertkonstruktoren (z.B. \texttt{Nil})
		\item Rekursionsfall: Konstruktoren mit Stelligkeit $>0$ (z.B. \texttt{Branch})
	\end{itemize}
	\pause
	
	Man \textbf{erzeugt} konkrete Bäume, indem man \texttt{Int} und \texttt{BinTree} aus der Typdefinition durch konkrete Werte ersetzt, z.B. \texttt{Int} durch \texttt{3} und \texttt{BinTree} durch \texttt{Nil} oder \texttt{Branch ...}.
	\pause
	
	\textbf{Pattern Matching} funktioniert weiterhin; man nutzt dafür die Wertkonstruktoren (hier: \texttt{Branch} und \texttt{Nil}):
	\begin{lstlisting}[style=bg]
foo :: BinTree -> ...
foo Nil = ...
foo (Branch x l r) = ...
	\end{lstlisting}
	\pause

	Um in GHCi eine \textbf{Ausgabe} der Bäume zu erhalten, muss \texttt{deriving Show} hinter die Typdefinition geschrieben werden. 
\end{frame}

\begin{frame}[t, fragile] \frametitle{Aufgabe 2 -- Teil (a)}
	\begin{lstlisting}[style=bg]
data BinTree = Branch Int BinTree BinTree | Nil
	\end{lstlisting}
	\pause
	
	Ein Beispielbaum: \\[6pt]
	\begin{lstlisting}[style=bg]
mytree :: BinTree 
mytree = Branch 0 
         ( Nil )
         ( Branch 3 
           ( Branch 1 Nil Nil )
           ( Branch 5 Nil Nil )
         )
	\end{lstlisting}
	... erfüllt die Suchbaumeigenschaft.
\end{frame}

\begin{frame}[t, fragile] \frametitle{Aufgabe 2 -- Teil (b)}
	\textbf{Test auf Baum-Gleichheit}
	
	\pause
	
	\begin{lstlisting}[style=bg]
data BinTree = Branch Int BinTree BinTree | Nil
equal :: BinTree -> BinTree -> Bool
	\end{lstlisting}
	
	\bigskip \pause
	
	\begin{lstlisting}[style=bg]
equal :: BinTree -> BinTree -> Bool
equal Nil              Nil              = True
equal Nil              (Branch y l2 r2) = False
equal (Branch x l1 r1) Nil              = False
equal (Branch x l1 r1) (Branch y l2 r2)
	= (x == y) && (equal l1 l2) && (equal r1 r2)
	\end{lstlisting}
\end{frame}

\begin{frame}[t, fragile] \frametitle{Aufgabe 2 -- Teil (c)}
	\textbf{Einfügen von Schlüsseln in einen Binärbaum}
	
	\begin{lstlisting}[style=bg]
data BinTree = Branch Int BinTree BinTree | Nil 
insert :: BinTree -> [Int] -> BinTree
	\end{lstlisting}
	
	\bigskip \pause
	
	\begin{lstlisting}[style=bg]
insert :: BinTree -> [Int] -> BinTree
insert t     [] = t
insert t (x:xs) = insert t' xs
    where t' = insertSingle t x
        insertSingle Nil            x = Branch x Nil Nil
        insertSingle (Branch y l r) x
            | x < y     = Branch y (insertSingle l x) r
            | otherwise = Branch y l (insertSingle r x)
	\end{lstlisting}
\end{frame}

\begin{frame}[fragile] \frametitle{Aufgabe 2 -- Teil (d)}
	\small
	\textbf{Levelorder-Traversierung}
	
	\begin{lstlisting}[style=bg]
data BinTree = Branch Int BinTree BinTree | Nil 
unwind :: BinTree -> [Int]
	\end{lstlisting}
	\smallskip \pause
	
	\textbf{Idee}: Nutze Liste von Bäumen als Zwischenspeicher (Hilfsfunktion) und hänge Kindbäume hinten an diese Liste an
	\bigskip \pause
	
	\begin{lstlisting}[style=bg, tabsize=1]
unwind :: BinTree -> [Int]
unwind t = go [t]
	where
		go []                  = []
		go ((Branch a Nil Nil) : ts) = a : go ts
		go ((Branch a l r)     : ts) = a : go (ts ++ [l,r])
	\end{lstlisting}
\end{frame}

\begin{frame} \frametitle{Ende}
	\centering
	\textbf{Fragen?}
\end{frame}

\end{document}