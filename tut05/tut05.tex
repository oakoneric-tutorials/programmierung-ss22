\documentclass{beamer}
\usepackage{../tut-slides}
\usepackage{../mathoperatorsAuD}

\usepackage{csquotes}
\usepackage{cancel}

\usepackage{amsmath,amssymb}


\usepackage{booktabs}
\usepackage{tabularx}
\usepackage{tabu}
\newcommand*\head{\rowfont{\bfseries}}
\newcommand*{\tw}{\rowfont{\ttfamily}}

\renewcommand{\tabularxcolumn}[1]{>{\hspace{0pt}}m{#1}}

\usepackage{listings}
\usepackage[skins]{tcolorbox}
\tcbuselibrary{listings}
%\lstset{
%	basicstyle=\footnotesize\ttfamily\color{black},
%	backgroundcolor=\color{white},
%	fillcolor=\color{white},
%	rulecolor=\color{gray},
%	tabsize=2,
%	% German umlauts
%	literate=%
%	{Ö}{{\"O}}1
%	{Ä}{{\"A}}1
%	{Ü}{{\"U}}1
%	{ß}{{\ss}}1
%	{ü}{{\"u}}1
%	{ä}{{\"a}}1
%	{ö}{{\"o}}1
%}
\newtcblisting{codebox}{%
	top=-6pt,
	bottom=-6pt,
	left=0pt,
	right=0pt,
	boxsep=5pt,
	enhanced,clip upper,
	colback=white,
	colframe=cddarkblue,
	fonttitle=\bfseries,
	listing only,
	listing options={basicstyle=\scriptsize\ttfamily}
}
\newtcblisting{smallcodebox}{%
	top=0pt,
	bottom=0pt,
	left=0pt,
	right=0pt,
	boxsep=5pt,
	enhanced,clip upper,
	colback=white,
	colframe=cddarkblue,
	listing only}

\lstdefinestyle{bg}{ 
	basicstyle=\footnotesize\ttfamily,        % the size of the fonts that are used for the code
	breakatwhitespace=false,         % sets if automatic breaks should only happen at whitespace
	breaklines=true,                 % sets automatic line breaking
	commentstyle=\itshape,    	     % comment style
	escapeinside={\%*}{\%*},          % if you want to add LaTeX within your code
	extendedchars=true,              % lets you use non-ASCII characters; for 8-bits encodings only, does not work with UTF-8
	firstnumber=1,                % start line enumeration with line 1000
	frame=L,
	backgroundcolor=\color{cdgray!10},
	%	keywordstyle=\bfseries,       % keyword style
	language=Haskell,                 % the language of the code
	numbers=left,                    % where to put the line-numbers; possible: (none, left, right)
	numbersep=8pt,                   % how far the line-numbers are from the code
	numberstyle=\tiny\color{cdgray!50}, % the style that is used for the line-numbers
	rulecolor=\color{cddarkblue}, 
	showstringspaces=false,
	tabsize=2,	                   % sets default tabsize to 2 spaces
}

\newcommand{\col}[1]{\textcolor{cdpurple}{#1}}
\newcommand{\step}[2][]{\ensuremath{\overset{{#1} (\text{#2})}{=}}}
\newcommand*{\astep}[2][]{\ensuremath{\overset{{#1} (\text{#2})}&{=}}}



\begin{document}	
	\title{Programmierung}
	\subtitle{Übung 5: Unifikation \& Induktion auf Listen}
	\author{Eric Kunze}
	\email{eric.kunze@tu-dresden.de}
	\city{TU Dresden}
	\date{09. Mai 2022}
%	\institute{Lehrstuhl für Grundlagen der Programmierung}
%	\titlegraphic{\includegraphics[width=2cm]{../TUD-white.pdf}}
	
	\maketitle
	

%%%%%%%%%%%%%%%%%%%%%%%%%%%%%%%%%%%%%%%%%%%%%%%%%%%%%%%%%%%%%%%%%%%%%%%%%%%%%

\begin{frame}[fragile] \frametitle{Inhalt}
	\begin{enumerate}
		\item Funktionale Programmierung
		\begin{enumerate}
			\item Einführung in Haskell
			\item Listen \& Algebraische Datentypen
			\item Funktionen höherer Ordnung
			\item Typpolymorphie \& \textbf{Unifikation}
			\item \textbf{Beweis von Programmeigenschaften}
			\item $\lambda$ -- Kalkül
		\end{enumerate}
		\item Logikprogrammierung
		\item Implementierung einer imperativen Programmiersprache
		\item Verifikation von Programmeigenschaften
		\item H${}_\text{0}$ -- ein einfacher Kern von Haskell
	\end{enumerate}
\end{frame}

%%%%%%%%%%%%%%%%%%%%%%%%%%%%%%%%%%%%%%%%%%%%%%%%%%%%%%%%%%%%%%%%%%%%%%%%%%%%%%%%%%%%%%%%%%%%%%%%%%%%%%%%%%%%%%%%%%%%%%%%%%%%%%%%%%%%%%%%%%%%%%%%%%%%%%%%%%%%%%%%%%%%%%%%%%%%%%%%%%%%%%%%%%%%%%%%%%%%%%%%%%%%%%%%%%%%%%%%%%%%%%%%%%%%%%%%%%%%%%%%%%%%%%%%%


\section{Unifikation \& Unifikationsalgorithmus \\ \textit{\normalsize Aufgabe 1}}

\begin{frame}[fragile, t] \frametitle{Erinnerung: Typpolymorphie}
	\footnotesize
	\begin{itemize}
		\item \textbf{bisher}: Funktionen mit konkreten Datentypen \\
		z.B. \texttt{length :: [Int] -> Int}
		\item \textbf{Problem}: Funktion würde auch auf anderen Datentypen funktionieren \\
		z.B. \texttt{length :: [Bool] -> Int} oder \texttt{length :: String -> Int}
		\item \textbf{Lösung}: Typvariablen und polymorphe Funktionen \\
		z.B. \texttt{length :: [\alert{a}] -> Int}
	\end{itemize}
	
	\pause
	
	Bei konkreter Instanziierung wird Typvariable an entsprechenden Typbezeichner gebunden (z.B. \texttt{\alert{a} = Int} oder \texttt{\alert{a} = Bool}).
	
	\begin{itemize}
		\item Der Aufruf \texttt{length [1,5,2,7]} liefert für die Typvariable \texttt{\alert{a} = Int}.
		\item Der Aufruf \texttt{length [True, False, True, True, False]} liefert die Belegung \texttt{\alert{a} = Bool}.
		\item Der Aufruf \texttt{length "hello"} impliziert \texttt{\alert{a} = Char}.
	\end{itemize}
\end{frame}


\begin{frame}[fragile] \frametitle{Unifikation}
	\footnotesize
	\textbf{Motivation:} Typüberprüfung
		 
	\begin{codebox}
f :: (t, Char) -> (t, [Char])
f (...) = ... 

g :: (Int, [u]) -> Int
g (...) = ...

h = g . f
	\end{codebox}
	
	\pause
	
	Wie müssen die Typvariablen \texttt{t} und \texttt{u} belegt werden, damit die Funktion \texttt{h} wohldefiniert ist, d.h. damit die Ergebnisse aus \texttt{f} wirklich in \texttt{g} eingesetzt werden dürfen?
\end{frame}

\begin{frame}[fragile] \frametitle{Typausdruck $\to$ Typterm}
	\footnotesize
	
	\textbf{Ziel:} theoretischere Form von Typausdrücken
	
	\textbf{Typausdrücke}
	\begin{itemize}
		\item \texttt{Int}, \texttt{Bool}, \texttt{Float}, \texttt{Char}, \texttt{String}
		\item Typvariablen
		\item Listen, Tupel, Funktionen
	\end{itemize}

	\textbf{Typterme}
	\begin{itemize}
		\item Übersetzung \textit{trans}: Typausdruck $\to$ Typterm
		\pause
		\item z.B.
		\begin{equation*}
			\begin{aligned}
				trans(\texttt{(t,[Char])}) &= ()^2(t, [](Char)) \\
				trans(\texttt{(Int, [u])}) &= ()^2(Int, [](u))
			\end{aligned}
		\end{equation*}
	\end{itemize}

	\pause
	
	Beide Typausdrücke können in Übereinstimmung gebracht werden, wenn die Typterme $trans(\texttt{(t,[Char])})$ und $trans(\texttt{(Int, [u])})$ unifizierbar sind.
	\\ \pause
	$\to$ \texttt{t = Int} und \texttt{u = Char}
\end{frame}

\begin{frame}\frametitle{Unifikationsalgorithmus}
	\footnotesize
	\begin{itemize}
		\item \textbf{gegeben:} zwei Typ\textit{terme} $t_1$, $t_2$ 
		\item \textbf{Ziel:} entscheide, ob $t_1$ und $t_2$ unifizierbar sind
	\end{itemize}
	
	Wir notieren die beiden Typterme als Spalte:
	\begin{equation*}
		\begin{pmatrix}
			t_1 \\ t_2
		\end{pmatrix} 
		\qquad \bzw \qquad 
		\begin{pmatrix}
			()^2(t, [](Char)) \\ ()^2(Int, [](u))
		\end{pmatrix} 
	\end{equation*}
	
	Unifikationsalgorithmus erstellt eine Folge von Mengen $M_i$, wobei die $M_{i+1}$ aus $M_i$ hervorgeht, indem eine der vier Regeln angewendet wird.
	\begin{equation*}
		M_1 \defeq \left\{ \begin{pmatrix}
			t_1 \\ t_2
		\end{pmatrix} \right\}
		\qquad \bzw \qquad 
		M_1 \defeq \left\{ \begin{pmatrix}
			()^2(t, [](Char)) \\ ()^2(Int, [](u))
		\end{pmatrix} \right\}
	\end{equation*}
\end{frame}

\begin{frame}\frametitle{Unifikationsalgorithmus -- Regeln}
	\footnotesize
	\begin{itemize}
		\item \textbf{Dekomposition.} Sei $\delta \in \Sigma$ ein $k$-stelliger Konstruktor, $s_1 , \dots , s_k , t_1 , \dots , t_k$ Terme über Konstruktoren und Variablen.
		\begin{equation*}
			\begin{pmatrix}
				\delta(s_1, \dots , s_k) \\ \delta(t_1, \dots , t_k)
			\end{pmatrix}
			\quad \leadsto \quad
			\begin{pmatrix} s_1 \\ t_1 \end{pmatrix} , \dots , \begin{pmatrix} s_k \\ t_k \end{pmatrix}
		\end{equation*}
		\item \textbf{Elimination.} Sei $x$ eine \alert{Variable} !
		\begin{equation*}
			\begin{pmatrix} x \\ x \end{pmatrix}
			\quad \leadsto \quad
			\emptyset
		\end{equation*}
		\item \textbf{Vertauschung.} Sei $t$ \alert{keine} Variable.
		\begin{equation*}
			\begin{pmatrix} t \\ x \end{pmatrix}
			\quad \leadsto \quad 
			\begin{pmatrix} x \\ t \end{pmatrix}
		\end{equation*}
		\item \textbf{Substitution.} Sei $x$ eine Variable, $t$ keine Variable. \\
		\textit{Occur Check:} $x$ kommt nicht in $t$ vor \\
		Dann ersetze in jedem anderen Term die Variable $x$ durch $t$.
		\begin{equation*}
			\begin{pmatrix} \alert{x} \\ t \end{pmatrix},
			\begin{pmatrix} y \\ s(\alert{x}) \end{pmatrix}
			\quad \leadsto \quad 
			\begin{pmatrix} x \\ t \end{pmatrix},			\begin{pmatrix} y \\ s(\alert{t}) \end{pmatrix}
		\end{equation*}
	\end{itemize}
\end{frame}

\begin{frame} \frametitle{Unifikationsalgorithmus}
	\footnotesize
	\textbf{Ende:} keine Regel mehr anwendbar -- Entscheidung:
	\begin{itemize}
		\item $t_1$, $t_2$ \textbf{unifizierbar}: $M$ ist von der Form
		\begin{equation*}
			\menge{\binom{u_1}{t_1}, \binom{u_2}{t_2}, \dots, \binom{u_k}{t_k}} \qquad 
			\begin{array}{c} 
				\footnotesize \text{\textit{\enquote{Variablen}}} \\ 
				\footnotesize \text{\textit{\enquote{Terme ohne Variablen}}}
			\end{array}
		\end{equation*}
		wobei $u_1, u_2, \dots, u_k$ paarweise verschiedene Variablen sind und nicht in $t_1, t_2, \dots, t_k$ vorkommen.  \\
		\textbf{allgemeinster Unifikator} $\phi$: 
		\begin{equation*}
			\begin{aligned}
				\phi(u_i) &= t_i & (i = 1, \dots, k) \\
				\phi(x) &= x     & \footnotesize \text{für alle nicht vorkommenden Variablen}
			\end{aligned}
		\end{equation*}
		\item $t_1$, $t_2$ sind \textbf{nicht unfizierbar}: $M$ hat nicht diese Form und keine Regel ist anwendbar
	\end{itemize}
	
	Weitere Unifikatoren $\psi$ erhält man durch Anwendung einer Substitution $\sigma$, sodass $\psi = \sigma \circ \phi$.
\end{frame}

\begin{frame} \frametitle{Occur Check}
	\footnotesize
	Um endlose Rekursionen zu unterbinden, benötigen die Regeln zum Vertauschen und zur Substitution gewisse Einschränkungen. 
		
	\textbf{Occur Check:}
	Gegeben sei ein Termpaar $\binom{x}{t}$, wobei $x$ eine Variable und $t$ ein Typterm sei.
	\vspace{-.5\baselineskip}
	\begin{itemize} 
		\item Kommt $x$ in $t$ vor, dann schlägt der Check fehl.
		\item Kommt $x$ nicht in $t$ vor, dann ist der Check okay.
	\end{itemize}
	\pause
	
	\textbf{Beispiel:}
	\begin{itemize}
		\item $\displaystyle\binom{x_1}{\gamma(x_1)} \leadsto$ Fehlschlag, da $x_1$ in $\gamma(x_1)$ vorkommt
		\item $\displaystyle\binom{x_1}{\gamma(x_2)} \leadsto$ okay, da $x_1$ nicht in $\gamma(x_2)$ vorkommt
	\end{itemize}
	\pause	

	Was passiert, wenn wir substituieren obwohl der Occur Check fehlschlägt?
	\begin{equation*}
		\binom{x_1}{\gamma(x_1)}, \binom{x_2}{\gamma(\alert{x_1})}
		\follows
		\binom{x_1}{\gamma(x_1)}, \binom{x_2}{\gamma(\alert{\gamma(x_1)})}
		\follows
		\binom{x_1}{\gamma(x_1)}, \binom{x_2}{\gamma(\gamma(\alert{\gamma(x_1)})}
	\end{equation*}
\end{frame}


\begin{frame} \frametitle{Aufgabe 1}
	\scriptsize
	\begin{align*}
		&\left\{\left(\begin{array}{lcrlcl}
		\col{\sigma(}\sigma(&x_1&,\alpha) \col{,} &\ \sigma(&\gamma(x_3) &, x_3)\col{)} \\
		\col{\sigma(}\sigma( &\gamma(x_2)&,\alpha) \col{,} &\ \sigma(&x_2 &, x_3)\col{)}
		\end{array}\right)\right\}\\
		%
		\overset{\text{Dek.}}{\Longrightarrow}
		%
		&\left\{\left(\begin{array}{lcr}
			\col{\sigma(} & x_1         & \col{,} \alpha\col{)} \\
			\col{\sigma(} & \gamma(x_2) & \col{,} \alpha\col{)}
		\end{array}\right),
		\left(\begin{array}{lcr}
			\col{\sigma(}& \gamma(x_3)  & \col{,} x_3)\col{)} \\
			\col{\sigma(}& x_2          & \col{,} x_3)\col{)}
		\end{array}\right)\right\} \\
		\overset{\text{Dek.}}{\Longrightarrow^2}
		%
		&\left\{ 
		\begin{pmatrix}
			x_1 \\ \gamma(x_2)
		\end{pmatrix} , \begin{pmatrix}
			\alpha \\ \alpha
		\end{pmatrix} , \begin{pmatrix}
			\gamma(x_3) \\ x_2
		\end{pmatrix} , \col{\begin{pmatrix}
			x_3 \\ x_3 
		\end{pmatrix}} 
		\right\} \\
		\overset{\text{El.}}{\Longrightarrow}
		%
		&\left\{
		\begin{pmatrix}
			x_1 \\ \gamma(x_2)
		\end{pmatrix} , 
		\col{\begin{pmatrix}
			\alpha \\ \alpha
		\end{pmatrix}} , 
		\begin{pmatrix}
			\gamma(x_3) \\ x_2
		\end{pmatrix}
		\right\} \\
		\overset{\text{Dek.}}{\Longrightarrow}
		%
		&\left\{
		\begin{pmatrix}
		x_1 \\ \gamma(x_2)
		\end{pmatrix} , \col{\begin{pmatrix}
			\gamma(x_3) \\ x_2
			\end{pmatrix}}
		\right\} \\
		\overset{\text{Vert.}}{\Longrightarrow}
		%
		&\left\{
		\begin{pmatrix}
		x_1 \\ \gamma(\col{x_2})
		\end{pmatrix} , \begin{pmatrix}
		\col{x_2} \\ \gamma(x_3)
		\end{pmatrix}
		\right\} \quad {\textcolor{cdgray}{x_2 \text{ kommt nicht in } \gamma(x_3) \text{ vor}}}\\
		%
		\overset{\text{Subst.}}{\Longrightarrow}
		&\left\{
		\begin{pmatrix}
		x_1 \\ \gamma(\gamma(x_3))
		\end{pmatrix} , \begin{pmatrix}
		x_2 \\ \gamma(x_3)
		\end{pmatrix}
		\right\}
	\end{align*}
\end{frame}

\begin{frame} \frametitle{Aufgabe 1}
	\footnotesize
	(a) \textbf{allgemeinster Unifikator:}
	\begin{align*}
		\qquad x_1 \mapsto \gamma(\gamma(x_3)) \qquad
		x_2 \mapsto \gamma(x_3) \qquad
		x_3 \mapsto x_3
	\end{align*}
	
	\pause
	
	(b) \textbf{weitere Unifikatoren:} 
	\begin{align*}
		x_1 &\mapsto \gamma(\gamma(\alpha))
		&x_2 &\mapsto \gamma(\alpha)
		&x_3 &\mapsto \alpha \\
		x_1 &\mapsto \gamma(\gamma(\gamma(\alpha)))
		&x_2 &\mapsto \gamma(\gamma(\alpha))
		&x_3 &\mapsto \gamma(\alpha)
	\end{align*}
	
	\pause
	
	(c) \textbf{Fehlschlag beim occur-check:}  \\ 
	\uncover<0>{(c)} \textcolor{cdgray}{Alphabet: $\Sigma = \left\{\gamma^{(1)} \right\}$}
	\begin{align*}
		t_1 &= x_1 \\
		t_2 &= \gamma(x_1)
	\end{align*}
\end{frame}

\begin{frame} \frametitle{Aufgabe 1 --- Teil (d)}
	\footnotesize
	\begin{align*}
		t_1 &= \texttt{(a , [a])} \\
		t_2 &= \texttt{(Int , [Double])} \\
		t_3 &= \texttt{(b , c)}
	\end{align*}
	
	\pause
	
	\begin{itemize}
		\item $t_1$ und $t_2$ sind \uncover<3->{\textit{nicht} unifizierbar}
		
		\item $t_1$ und $t_3$ sind \uncover<4->{unifizierbar mit $\texttt{a} \mapsto \texttt{a}$, $\texttt{b} \mapsto \texttt{a}$, $\texttt{c} \mapsto \texttt{[a]}$}
		
		\item $t_2$ und $t_3$ sind \uncover<5->{unifizierbar mit $\texttt{b} \mapsto \texttt{Int}$, $\texttt{c} \mapsto \texttt{[Double]}$}
	\end{itemize}
\end{frame}



\section{Induktionsbeweise \\ \textit{\normalsize Aufgaben 2}}

\begin{frame} \frametitle{Vollständige Induktion auf $\N$}
	\footnotesize
	
	\textbf{Definition:} natürliche Zahlen $\N \defeq \menge{0, 1, \dots}$
	\begin{equation*}
		\begin{aligned}
			\text{Basisfall: } \quad &&0 &\in \N \\
			\text{Rekursionsfall: }\quad &&x + 1 &\in \N
			\quad \text{ für } \quad x \in \N
		\end{aligned}
	\end{equation*}
	
	\textbf{Beweis von Eigenschaften:}
	Eigenschaft $=$ Prädikat $P$
	\begin{center}
		\fbox{zu zeigen: \hspace{2em} für alle $x \in \N$ gilt $P(x)$}
	\end{center}
	
	\textbf{vollständige Induktion:}
	\begin{itemize}
		\item \textcolor{cdblue}{Induktionsanfang}: \\
		zeige $P(x)$ für $x = 0$
		\item \textcolor{cdblue}{Induktionsvoraussetzung}:\\
		Sei $x \in \N$, sodass $P(x)$ gilt.
		\hfill
		\textcolor{cdgray}{$P(x)$ gilt noch nicht für \textit{alle} $x \in \N$}
		\item \textcolor{cdblue}{Induktionsschritt}: \\
		zeige $P(x+1)$ unter Nutzung der Induktionsvoraussetzung
	\end{itemize}
\end{frame}

\begin{frame} \frametitle{Induktion auf Listen}
	\footnotesize
	
	\textbf{Erinnerung:} Rekursion über Listen \texttt{xs}
	\begin{equation*}
		\begin{aligned}
			\text{Basisfall: } \quad \texttt{xs} &= \texttt{[]} \\
			\text{Rekursionsfall: }\quad \texttt{xs} &= \texttt{(y:ys)} \quad \text{ für } \quad \texttt{ys :: [a]}
		\end{aligned}
	\end{equation*}
	\pause
	
	\textbf{Beweis von Programmeigenschaften:}
	Eigenschaft $=$ Prädikat $P$
	\begin{center}
		\fbox{zu zeigen: \hspace{2em} für alle \texttt{xs :: [a]} gilt $P(\texttt{xs})$}
	\end{center}
	\pause
	
	\textbf{Induktion auf Listen:} \vspace{-0.5\baselineskip}
	\begin{itemize}
		\item \textcolor{cdblue}{Induktionsanfang}: \\
		zeige $P(\texttt{xs})$ für \texttt{xs == []}
		\pause
		\item \textcolor{cdblue}{Induktionsvoraussetzung}:\\
		Sei \texttt{xs :: [a]} eine Liste für die $P(\texttt{xs})$ gilt. \pause
		\item \textcolor{cdblue}{Induktionsschritt}: \\
		zeige $P(\texttt{x:xs})$ für alle \texttt{x :: a} unter Nutzung der Induktionsvoraussetzung
		\pause
	\end{itemize}
	
	\fbox{\emph{Allgemeiner Hinweis:} Es müssen immer \alert{alle} Variablen quantifiziert werden!}
\end{frame}

\begin{frame} \frametitle{Strukturelle Induktion}
	\footnotesize
	
	\textbf{Erinnerung:} Rekursion über Bäume
	\begin{equation*}
		\begin{aligned}
			\text{Basisfall: } \quad &\texttt{Nil} \text{ oder } \texttt{Leaf x} \text{ für } \texttt{x :: a} \\
			\text{Rekursionsfall: }\quad &\texttt{Branch x l r} \text{ für } \texttt{x :: a} \text{ und } \texttt{l,r :: BinTree a}
		\end{aligned}
	\end{equation*}
	\pause
	
	\begin{center}
		\fbox{zu zeigen: \hspace{2em} für alle \texttt{t :: BinTree a} gilt $P(\texttt{t})$}
	\end{center}
	\pause
	
	\textbf{strukturelle Induktion:} \vspace{-0.5\baselineskip}
	\begin{itemize}
		\item \textcolor{cdblue}{Induktionsanfang}: \\
		zeige $P(\texttt{t})$ für \texttt{t == Nil} oder \texttt{t == Leaf x} für alle \texttt{x :: a}
		\pause
		\item \textcolor{cdblue}{Induktionsvoraussetzung}:\\
		Seien \texttt{l, r :: BinTree a} zwei Bäume, sodass $P(\texttt{l})$ und $P(\texttt{r})$ gilt. \pause
		\item \textcolor{cdblue}{Induktionsschritt}: \\
		zeige $P(\texttt{Branch x l r})$ für alle \texttt{x :: a} unter Nutzung der Induktionsvoraussetzung
		\pause
	\end{itemize}
	
	\fbox{\emph{Allgemeiner Hinweis:} Es müssen immer \alert{alle} Variablen quantifiziert werden!}
\end{frame}

\begin{frame} \frametitle{Fehlerquellen}
	\footnotesize
	\begin{itemize}
		\item kein Induktionsprinzip
		\item IV wird im Induktionsschritt nicht verwendet
		\item fehlende Quantifizierung (nur Gleichungen bringen kaum Punkte)
		\item \textit{Missachtung freier Variablen} \pause
		\item zu beweisende Eigenschaft $P$ wird für \texttt{xs} angenommen, um sie dann im Induktionsschritt nochmal für \texttt{xs} zu beweisen –- eine Tautologie
		\item Annahme, dass $P$ bereits für alle Listen gilt, um es dann für \texttt{x:xs} nochmal zu zeigen		
	\end{itemize}
\end{frame}

\begin{frame} \frametitle{Aufgabe 2}
	\scriptsize
	Zu zeigen ist die Gleichung
	\begin{equation*}
		\texttt{sum (foo xs) = 2 * sum xs - length xs} \qquad \text{für alle } \texttt{xs :: Int}
	\end{equation*}
	mittels Induktion über Listen.
	
	\textbf{Induktionsanfang:} Sei \texttt{xs == []}. 
	\begin{align*}
		\intertext{linke Seite: }
		\texttt{sum (foo [])} &\step{2} \texttt{sum [] \step{6} 0} \\
		\intertext{rechte Seite: }
		\texttt{2 * sum [] - length []} &\step{10} \texttt{2 * sum [] - 0 \step{6} 2 * 0 - 0 = 0}
	\end{align*}
	
	\textbf{Induktionsvoraussetzung:} Sei \texttt{xs :: [Int]}, sodass
	\begin{equation*}
		\texttt{sum (foo xs) = 2 * sum xs - length xs}
	\end{equation*}
	gilt.
\end{frame}

\begin{frame} \frametitle{Aufgabe 2 (Fortsetzung)}
	\scriptsize
	
	\textbf{Induktionsschritt:} Sei \texttt{x :: Int}. Es gilt
	\begin{align*}
		\texttt{sum (foo (x:xs))} \astep{3} \texttt{sum (x : x : (-1) : foo xs)} \\
		\astep[3*]{7} \texttt{x + x + (-1) + sum (foo xs)} \\
		\astep{IV} \texttt{x + x + (-1) + 2 * sum xs - length xs} \\
		\astep{Komm.} \texttt{2 * x + 2 * sum xs - 1 - length xs)} \\
		\astep{Dist.} \texttt{2 * (x + sum xs) - (1 + length xs)} \\
		\astep{7} \texttt{2 * sum (x:xs) - (1 + length xs)} \\
		\astep{11} \texttt{2 * sum (x:xs) - length (x:xs)} 
	\end{align*}
\end{frame}

%\begin{frame} \frametitle{Aufgabe 2 --- Teil (a)}
%	\scriptsize
%	Sei \texttt{a} ein beliebiger Typ. Zu zeigen ist die Gleichung
%	\begin{equation}
%		\begin{aligned}
%			\texttt{[x] ++ rev ys ++ rev xs = rev (xs ++ ys ++ [x])} \\ \text{ für alle } \texttt{xs, ys :: [a]}
%		\end{aligned}
%		\tag{H3}
%	\end{equation}
%	
%	Der Beweis funktioniert ohne Induktion:
%	\begin{align*}
%		\texttt{[x] ++ rev ys ++ rev xs}
%		\astep{Ass.} 
%		\texttt{[x] ++ (rev ys ++ rev xs)} \\
%		\astep{H2} \texttt{[x] ++ rev (xs ++ ys)} \\
%		\astep{H1} \texttt{rev [x] ++ rev (xs ++ ys)} \\
%		\astep{H2} \texttt{rev ((xs ++ ys) ++ [x])} \\
%		\astep{Ass.} \texttt{rev (xs ++ ys ++ [x])}
%	\end{align*}
%\end{frame}
%
%\begin{frame} \frametitle{Aufgabe 2 --- Teil (b)}
%	\scriptsize
%	Sei \texttt{a} ein beliebiger Typ. Zu zeigen ist die Gleichung
%	\begin{equation*}
%		\texttt{preOrder t = rev (mPostOrder t)} \qquad \text{für alle} \qquad \texttt{t :: BinTree a}
%	\end{equation*}
%	mittels struktureller Induktion.
%	
%	\textbf{Induktionsanfang:} Sei \texttt{x :: a} beliebig und \texttt{t = Leaf x}. 
%	\begin{align*}
%		\text{linke Seite: } \quad &\texttt{preOrder (Leaf x)} \step{4} \texttt{[x]} \\
%		\text{rechte Seite: } \quad & \texttt{rev (mPostOrder (Leaf x))} \step{8} \texttt{rev [x] \step{H1} [x]}
%	\end{align*}
%	
%	\textbf{Induktionsvoraussetzung:} Seien \texttt{l, r:: BinTree a}, sodass
%	\begin{align}
%		\texttt{preOrder l} &= \texttt{rev(mPostOrder l)} \tag{IV1} \\
%		\texttt{preOrder r} &= \texttt{rev(mPostOrder r)} \tag{IV2}
%	\end{align}
%	gilt.
%\end{frame}
%
%\begin{frame} \frametitle{Aufgabe 2 -- Teil (b) (Fortsetzung)}
%	\scriptsize
%	
%	\textbf{Induktionsschritt:} Sei \texttt{x :: a} beliebig. Es gilt
%	\begin{align*}
%		\texttt{preOrder (Node x l r)} \astep{5} \texttt{[x] ++ preOrder l ++ preOrder r} \\
%		\astep{IV1} \texttt{[x] ++ rev(mPostOrder l) ++ preOrder r} \\
%		\astep{IV2} \texttt{[x] ++ rev(mPostOrder l) ++ rev(mPostOrder r)} \\
%		\astep{H3} \texttt{rev(mPostOrder r ++ mPostOrder l ++ [x])} \\
%		\astep{9} \texttt{rev(mPostOrder (Node x l r))}
%	\end{align*}
%\end{frame}

\begin{frame} \frametitle{Ende}
	\centering
	\textbf{Fragen?}
\end{frame}

\end{document}

