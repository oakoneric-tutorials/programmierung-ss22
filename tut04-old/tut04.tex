\documentclass{beamer}
\usepackage{../tut-slides}
\usepackage{../mathoperatorsAuD}

\usepackage{csquotes}
\usepackage{cancel}

\usepackage{amsmath,amssymb}


\usepackage{booktabs}
\usepackage{tabularx}
\usepackage{tabu}
\newcommand*\head{\rowfont{\bfseries}}
\newcommand*{\tw}{\rowfont{\ttfamily}}

\renewcommand{\tabularxcolumn}[1]{>{\hspace{0pt}}m{#1}}

\usepackage{listings}
\usepackage[skins]{tcolorbox}
\tcbuselibrary{listings}
%\lstset{
%	basicstyle=\footnotesize\ttfamily\color{black},
%	backgroundcolor=\color{white},
%	fillcolor=\color{white},
%	rulecolor=\color{gray},
%	tabsize=2,
%	% German umlauts
%	literate=%
%	{Ö}{{\"O}}1
%	{Ä}{{\"A}}1
%	{Ü}{{\"U}}1
%	{ß}{{\ss}}1
%	{ü}{{\"u}}1
%	{ä}{{\"a}}1
%	{ö}{{\"o}}1
%}
\newtcblisting{codebox}{%
	top=-6pt,
	bottom=-6pt,
	left=0pt,
	right=0pt,
	boxsep=5pt,
	enhanced,clip upper,
	colback=white,
	colframe=cddarkblue,
	fonttitle=\bfseries,
	listing only,
	listing options={basicstyle=\scriptsize\ttfamily}
}
\newtcblisting{smallcodebox}{%
	top=0pt,
	bottom=0pt,
	left=0pt,
	right=0pt,
	boxsep=5pt,
	enhanced,clip upper,
	colback=white,
	colframe=cddarkblue,
	listing only}

\newtcblisting{commandshell}{%
	colback=black,colupper=white,colframe=yellow!75!black,
	listing only,listing options={style=tcblatex,language=sh},
	every listing line={\textcolor{red}{\small\ttfamily\bfseries root \$> }}}

\newcommand{\col}[1]{\textcolor{cdpurple}{#1}}


\begin{document}	
	\title{Programmierung}
	\subtitle{Übung 4: Typpolymorphie \& Unifikation}
	\author{Eric Kunze}
	\email{eric.kunze@mailbox.tu-dresden.de}
%	\city{TU Dresden}
	\date{}
%	\institute{Lehrstuhl für Grundlagen der Programmierung}
%	\titlegraphic{\includegraphics[width=2cm]{../TUD-white.pdf}}
	
	\maketitle
	

%%%%%%%%%%%%%%%%%%%%%%%%%%%%%%%%%%%%%%%%%%%%%%%%%%%%%%%%%%%%%%%%%%%%%%%%%%%%%

\begin{frame}[fragile] \frametitle{Inhalt}
	\begin{enumerate}
		\item Funktionale Programmierung
		\begin{enumerate}
			\item Einführung in Haskell
			\item Listen \& Algebraische Datentypen
			\item Funktionen höherer Ordnung
			\item \textbf{Typpolymorphie \& Unifikation}
			\item Beweis von Programmeigenschaften
			\item $\lambda$ -- Kalkül
		\end{enumerate}
		\item Logikprogrammierung
		\item Implementierung einer imperativen Programmiersprache
		\item Verifikation von Programmeigenschaften
		\item H${}_\text{0}$ -- ein einfacher Kern von Haskell
	\end{enumerate}
\end{frame}



\section{Typpolymorphie \\ \textit{\normalsize Aufgabe 1}}

\begin{frame}[fragile, t] \frametitle{Typpolymorphie}
	\footnotesize
	\begin{itemize}
		\item \textbf{bisher}: Funktionen mit konkreten Datentypen \\
		z.B. \texttt{length :: [Int] -> Int}
		\item \textbf{Problem}: Funktion würde auch auf anderen Datentypen funktionieren \\
		z.B. \texttt{length :: [Bool] -> Int} oder \texttt{length :: String -> Int}
		\item \textbf{Lösung}: Typvariablen und polymorphe Funktionen \\
		z.B. \texttt{length :: [\alert{a}] -> Int}
	\end{itemize}
	
	\pause
	
	Bei konkreter Instanziierung wird Typvariable an entsprechenden Typbezeichner gebunden (z.B. \texttt{\alert{a} = Int} oder \texttt{\alert{a} = Bool}).
	
	\begin{itemize}
		\item Der Aufruf \texttt{length [1,5,2,7]} liefert für die Typvariable \texttt{\alert{a} = Int}.
		\item Der Aufruf \texttt{length [True, False, True, True, False]} liefert die Belegung \texttt{\alert{a} = Bool}.
		\item Der Aufruf \texttt{length "hello"} impliziert \texttt{\alert{a} = Char}.
	\end{itemize}
\end{frame}


\begin{frame}[fragile, t] \frametitle{Aufgabe 1 --- Teil (a)}
	\textbf{Beispielbaum mit mindestens 5 Blättern} \\[6pt]
	\begin{codebox}
data BinTree a = Branch a (BinTree a) (BinTree a) | Leaf a
	\end{codebox}
	\pause
	\begin{codebox}
testTree :: BinTree Int
testTree = Branch 5
             (Leaf 1)
             (Branch 12
                (Branch 4
                   (Leaf 0)
                   (Leaf 0))
                (Branch 12
                   (Leaf 0)
                   (Leaf 1)))
	\end{codebox}
\end{frame}

\begin{frame}[fragile, t] \frametitle{Aufgabe 1 --- Teil (b)}
	\textbf{minimale Tiefe eines Baumes}
	
	\begin{codebox}
data BinTree a = Branch a (BinTree a) (BinTree a) | Leaf a
depth :: BinTree a -> Int
	\end{codebox}
	
	\bigskip \pause
	
	\begin{codebox}
depth :: BinTree a -> Int
depth (Leaf   _    ) = 1
depth (Branch _ l r) = 1 + min (depth l) (depth r)
	\end{codebox}
\end{frame}


\begin{frame}[fragile, t] \frametitle{Aufgabe 1 --- Teil (c)}
	\textbf{Baum mit Beschriftungsfolge neu beschriften}
	
	\begin{codebox}
data BinTree a = Branch a (BinTree a) (BinTree a) | Leaf a
paths :: BinTree a -> BinTree [a]
	\end{codebox}
	
	\bigskip \pause
	
	\begin{codebox}
paths :: BinTree a -> BinTree [a]
paths = go []
  where
      go :: [a] -> BinTree a -> BinTree [a]
      go prefix (Leaf   x    ) = Leaf (prefix ++ [x])
      go prefix (Branch x l r) 
      	= Branch (prefix ++ [x]) 
      	    (go (prefix ++ [x]) l) 
      	    (go (prefix ++ [x]) r)
	\end{codebox}
\end{frame}


\begin{frame}[fragile, t] \frametitle{Aufgabe 1 --- Teil (d)}
	\textbf{Map für Bäume}
	
	\begin{codebox}
data BinTree a = Branch a (BinTree a) (BinTree a) | Leaf a
tmap :: (a -> b) -> BinTree a -> BinTree b
	\end{codebox}
	
	\bigskip \pause
	
	\begin{codebox}
tmap :: (a -> b) -> BinTree a -> BinTree b
tmap f (Leaf   x    ) = Leaf   (f x)
tmap f (Branch x l r) = Branch (f x) (tmap f l) (tmap f r)
	\end{codebox}
\end{frame}


{
\setbeamercolor{frametitle}{fg=white, bg=cdgray}	
\color{cdgray}
\begin{frame}[fragile, t] \frametitle{Einschub: Auswertung eines Funktionsaufrufs}
	\begin{codebox}
map :: (a -> b) -> [a] -> [b]
map _ []       = []
map f (x : xs) = f x : map f xs

uncurry :: (a -> b -> c) -> (a, b) -> c
uncurry f (x, y) = f x y
	\end{codebox}
	
	\footnotesize
	\begin{ttfamily}
		map (uncurry (+)) [(1,2), (3,4)] \\ \pause
		= map (uncurry (+)) ((1,2):[(3,4)]) \\
		= uncurry (+) (1,2) : map (uncurry (+)) [(3,4)] \\
		= (1 + 2) : map (uncurry (+)) [(3,4)] \\
		= 3 : map (uncurry (+)) [(3,4)] \\
		= 3 : map (uncurry (+)) ((3,4);[]) \\
		= 3 : uncurry (+) (3,4) : map (uncurry (+)) [] \\
		= 3 : (3 + 4) : map (uncurry (+)) [] \\
		= 3 : 7 : map (uncurry (+)) [] \\
		= 3 : 7 : [] \\
		= [3, 7]
	\end{ttfamily}

\end{frame}
}

%%%%%%%%%%%%%%%%%%%%%%%%%%%%%%%%%%%%%%%%%%%%%%%%%%%%%%%%%%%%%%%%%%%%%%%%%%%%%%%%%%%%%%%%%%%%%%%%%%%%%%%%%%%%%%%%%%%%%%%%%%%%%%%%%%%%%%%%%%%%%%%%%%%%%%%%%%%%%%%%%%%%%%%%%%%%%%%%%%%%%%%%%%%%%%%%%%%%%%%%%%%%%%%%%%%%%%%%%%%%%%%%%%%%%%%%%%%%%%%%%%%%%%%%%


\section{Unifikation \& Unifikationsalgorithmus \\ \textit{\normalsize Aufgabe 2}}


\begin{frame}[fragile] \frametitle{Unifikation}
	\footnotesize
	\textbf{Motivation:} Typüberprüfung
		 
	\begin{codebox}
f :: (t, Char) -> (t, [Char])
f (...) = ... 

g :: (Int, [u]) -> Int
g (...) = ...

h = g . f
	\end{codebox}
	
	\pause
	
	Wie müssen die Typvariablen \texttt{t} und \texttt{u} belegt werden, damit die Funktion \texttt{h} wohldefiniert ist, d.h. damit die Ergebnisse aus \texttt{f} wirklich in \texttt{g} eingesetzt werden dürfen?
\end{frame}

\begin{frame}[fragile] \frametitle{Typausdruck $\to$ Typterm}
	\footnotesize
	
	\textbf{Ziel:} theoretischere Form von Typausdrücken
	
	\textbf{Typausdrücke}
	\begin{itemize}
		\item \texttt{Int}, \texttt{Bool}, \texttt{Float}, \texttt{Char}, \texttt{String}
		\item Typvariablen
		\item Listen, Tupel, Funktionen
	\end{itemize}

	\textbf{Typterme}
	\begin{itemize}
		\item Übersetzung \textit{trans}: Typausdruck $\to$ Typterm
		\pause
		\item z.B.
		\begin{equation*}
			\begin{aligned}
				trans(\texttt{(t,[Char])}) &= ()^2(t, [](Char)) \\
				trans(\texttt{(Int, [u])}) &= ()^2(Int, [](u))
			\end{aligned}
		\end{equation*}
	\end{itemize}

	\pause
	
	Beide Typausdrücke können in Übereinstimmung gebracht werden, wenn die Typterme $trans(\texttt{(t,[Char])})$ und $trans(\texttt{(Int, [u])})$ unifizierbar sind.
	\\ \pause
	$\to$ \texttt{t = Int} und \texttt{u = Char}
\end{frame}

\begin{frame}\frametitle{Unifikationsalgorithmus}
	\footnotesize
	\begin{itemize}
		\item \textbf{gegeben:} zwei Typ\textit{terme} $t_1$, $t_2$ 
		\item \textbf{Ziel:} entscheide, ob $t_1$ und $t_2$ unifizierbar sind
	\end{itemize}
	
	Wir notieren die beiden Typterme als Spalte:
	\begin{equation*}
		\begin{pmatrix}
			t_1 \\ t_2
		\end{pmatrix} 
		\qquad \bzw \qquad 
		\begin{pmatrix}
			()^2(t, [](Char)) \\ ()^2(Int, [](u))
		\end{pmatrix} 
	\end{equation*}
	
	Unifikationsalgorithmus erstellt eine Folge von Mengen $M_i$, wobei die $M_{i+1}$ aus $M_i$ hervorgeht, indem eine der vier Regeln angewendet wird.
	\begin{equation*}
		M_1 \defeq \left\{ \begin{pmatrix}
			t_1 \\ t_2
		\end{pmatrix} \right\}
		\qquad \bzw \qquad 
		M_1 \defeq \left\{ \begin{pmatrix}
			()^2(t, [](Char)) \\ ()^2(Int, [](u))
		\end{pmatrix} \right\}
	\end{equation*}
\end{frame}

\begin{frame}\frametitle{Unifikationsalgorithmus -- Regeln}
	\footnotesize
	\begin{itemize}
		\item \textbf{Dekomposition.} Sei $\delta \in \Sigma$ ein $k$-stelliger Konstruktor, $s_1 , \dots , s_k , t_1 , \dots , t_k$ Terme über Konstruktoren und Variablen.
		\begin{equation*}
			\begin{pmatrix}
				\delta(s_1, \dots , s_k) \\ \delta(t_1, \dots , t_k)
			\end{pmatrix}
			\quad \leadsto \quad
			\begin{pmatrix} s_1 \\ t_1 \end{pmatrix} , \dots , \begin{pmatrix} s_k \\ t_k \end{pmatrix}
		\end{equation*}
		\item \textbf{Elimination.} Sei $x$ eine \alert{Variable} !
		\begin{equation*}
			\begin{pmatrix} x \\ x \end{pmatrix}
			\quad \leadsto \quad
			\emptyset
		\end{equation*}
		\item \textbf{Vertauschung.} Sei $t$ \alert{keine} Variable.
		\begin{equation*}
			\begin{pmatrix} t \\ x \end{pmatrix}
			\quad \leadsto \quad 
			\begin{pmatrix} x \\ t \end{pmatrix}
		\end{equation*}
		\item \textbf{Substitution.} Sei $x$ eine Variable, $t$ keine Variable. \\
		\textit{Occur Check:} $x$ kommt nicht in $t$ vor \\
		Dann ersetze in jedem anderen Term die Variable $x$ durch $t$.
		\begin{equation*}
			\begin{pmatrix} \alert{x} \\ t \end{pmatrix},
			\begin{pmatrix} y \\ s(\alert{x}) \end{pmatrix}
			\quad \leadsto \quad 
			\begin{pmatrix} x \\ t \end{pmatrix},			\begin{pmatrix} y \\ s(\alert{t}) \end{pmatrix}
		\end{equation*}
	\end{itemize}
\end{frame}

\begin{frame} \frametitle{Unifikationsalgorithmus}
	\footnotesize
	\textbf{Ende:} keine Regel mehr anwendbar -- Entscheidung:
	\begin{itemize}
		\item $t_1$, $t_2$ \textbf{unifizierbar}: $M$ ist von der Form
		\begin{equation*}
			\menge{\binom{u_1}{t_1}, \binom{u_2}{t_2}, \dots, \binom{u_k}{t_k}} \qquad 
			\begin{array}{c} 
				\footnotesize \text{\textit{\enquote{Variablen}}} \\ 
				\footnotesize \text{\textit{\enquote{Terme ohne Variablen}}}
			\end{array}
		\end{equation*}
		wobei $u_1, u_2, \dots, u_k$ paarweise verschiedene Variablen sind und nicht in $t_1, t_2, \dots, t_k$ vorkommen.  \\
		\textbf{allgemeinster Unifikator} $\phi$: 
		\begin{equation*}
			\begin{aligned}
				\phi(u_i) &= t_i & (i = 1, \dots, k) \\
				\phi(x) &= x     & \footnotesize \text{für alle nicht vorkommenden Variablen}
			\end{aligned}
		\end{equation*}
		\item $t_1$, $t_2$ sind \textbf{nicht unfizierbar}: $M$ hat nicht diese Form und keine Regel ist anwendbar
	\end{itemize}
\end{frame}

\begin{frame} \frametitle{Occur Check}
	\footnotesize
	Um endlose Rekursionen zu unterbinden, benötigen die Regeln zum Vertauschen und zur Substitution gewisse Einschränkungen. 
		
	\textbf{Occur Check:}
	Gegeben sei ein Termpaar $\binom{x}{t}$, wobei $x$ eine Variable und $t$ ein Typterm sei.
	\vspace{-.5\baselineskip}
	\begin{itemize} 
		\item Kommt $x$ in $t$ vor, dann schlägt der Check fehl.
		\item Kommt $x$ nicht in $t$ vor, dann ist der Check okay.
	\end{itemize}

	\textbf{Beispiel:}
	\begin{itemize}
		\item $\displaystyle\binom{x_1}{\gamma(x_1)} \leadsto$ Fehlschlag, da $x_1$ in $\gamma(x_1)$ vorkommt
		\item $\displaystyle\binom{x_1}{\gamma(x_2)} \leadsto$ okay, da $x_1$ nicht in $\gamma(x_2)$ vorkommt
	\end{itemize}

	Was passiert, wenn wir substituieren obwohl der Occur Check fehlschlägt?
	\begin{equation*}
		\binom{x_1}{\gamma(x_1)}, \binom{x_2}{\gamma(\alert{x_1})}
		\follows
		\binom{x_1}{\gamma(x_1)}, \binom{x_2}{\gamma(\alert{\gamma(x_1)})}
		\follows
		\binom{x_1}{\gamma(x_1)}, \binom{x_2}{\gamma(\gamma(\alert{\gamma(x_1)})}
	\end{equation*}
\end{frame}


\begin{frame} \frametitle{Aufgabe 3}
	\scriptsize
	\begin{align*}
		&\left\{\left(\begin{array}{lcrlcl}
		\col{\sigma(}\sigma(&x_1&,\alpha) \col{,} &\ \sigma(&\gamma(x_3) &, x_3)\col{)} \\
		\col{\sigma(}\sigma( &\gamma(x_2)&,\alpha) \col{,} &\ \sigma(&x_2 &, x_3)\col{)}
		\end{array}\right)\right\}\\
		%
		\overset{\text{Dek.}}{\Longrightarrow}
		%
		&\left\{\left(\begin{array}{lcr}
			\col{\sigma(} & x_1         & \col{,} \alpha\col{)} \\
			\col{\sigma(} & \gamma(x_2) & \col{,} \alpha\col{)}
		\end{array}\right),
		\left(\begin{array}{lcr}
			\col{\sigma(}& \gamma(x_3)  & \col{,} x_3)\col{)} \\
			\col{\sigma(}& x_2          & \col{,} x_3)\col{)}
		\end{array}\right)\right\} \\
		\overset{\text{Dek.}}{\Longrightarrow^2}
		%
		&\left\{ 
		\begin{pmatrix}
			x_1 \\ \gamma(x_2)
		\end{pmatrix} , \begin{pmatrix}
			\alpha \\ \alpha
		\end{pmatrix} , \begin{pmatrix}
			\gamma(x_3) \\ x_2
		\end{pmatrix} , \col{\begin{pmatrix}
			x_3 \\ x_3 
		\end{pmatrix}} 
		\right\} \\
		\overset{\text{El.}}{\Longrightarrow}
		%
		&\left\{
		\begin{pmatrix}
			x_1 \\ \gamma(x_2)
		\end{pmatrix} , 
		\col{\begin{pmatrix}
			\alpha \\ \alpha
		\end{pmatrix}} , 
		\begin{pmatrix}
			\gamma(x_3) \\ x_2
		\end{pmatrix}
		\right\} \\
		\overset{\text{Dek.}}{\Longrightarrow}
		%
		&\left\{
		\begin{pmatrix}
		x_1 \\ \gamma(x_2)
		\end{pmatrix} , \col{\begin{pmatrix}
			\gamma(x_3) \\ x_2
			\end{pmatrix}}
		\right\} \\
		\overset{\text{Vert.}}{\Longrightarrow}
		%
		&\left\{
		\begin{pmatrix}
		x_1 \\ \gamma(\col{x_2})
		\end{pmatrix} , \begin{pmatrix}
		\col{x_2} \\ \gamma(x_3)
		\end{pmatrix}
		\right\} \quad {\textcolor{cdgray}{x_2 \text{ kommt nicht in } \gamma(x_3) \text{ vor}}}\\
		%
		\overset{\text{Subst.}}{\Longrightarrow}
		&\left\{
		\begin{pmatrix}
		x_1 \\ \gamma(\gamma(x_3))
		\end{pmatrix} , \begin{pmatrix}
		x_2 \\ \gamma(x_3)
		\end{pmatrix}
		\right\}
	\end{align*}
\end{frame}

\begin{frame} \frametitle{Aufgabe 3}
	\footnotesize
	(a) \textbf{allgemeinster Unifikator:}
	\begin{align*}
		\qquad x_1 \mapsto \gamma(\gamma(x_3)) \qquad
		x_2 \mapsto \gamma(x_3) \qquad
		x_3 \mapsto x_3
	\end{align*}
	
	\pause
	
	(b) \textbf{weitere Unifikatoren:} 
	\begin{align*}
		x_1 &\mapsto \gamma(\gamma(\alpha))
		&x_2 &\mapsto \gamma(\alpha)
		&x_3 &\mapsto \alpha \\
		x_1 &\mapsto \gamma(\gamma(\gamma(\alpha)))
		&x_2 &\mapsto \gamma(\gamma(\alpha))
		&x_3 &\mapsto \gamma(\alpha)
	\end{align*}
	
	\pause
	
	(c) \textbf{Fehlschlag beim occur-check:}  \\ 
	\uncover<0>{(c)} \textcolor{cdgray}{Alphabet: $\Sigma = \left\{\gamma^{(1)} \right\}$}
	\begin{align*}
		t_1 &= x_1 \\
		t_2 &= \gamma(x_1)
	\end{align*}
\end{frame}

\begin{frame} \frametitle{Aufgabe 3 --- Teil (d)}
	\footnotesize
	\begin{align*}
		t_1 &= \texttt{(a , [a])} \\
		t_2 &= \texttt{(Int , [Double])} \\
		t_3 &= \texttt{(b , c)}
	\end{align*}
	
	\pause
	
	\begin{itemize}
		\item $t_1$ und $t_2$ sind \uncover<3->{\textit{nicht} unifizierbar}
		
		\item $t_1$ und $t_3$ sind \uncover<4->{unifizierbar mit $\texttt{a} \mapsto \texttt{a}$, $\texttt{b} \mapsto \texttt{a}$, $\texttt{c} \mapsto \texttt{[a]}$}
		
		\item $t_2$ und $t_3$ sind \uncover<5->{unifizierbar mit $\texttt{b} \mapsto \texttt{Int}$, $\texttt{c} \mapsto \texttt{[Double]}$}
	\end{itemize}
\end{frame}


\begin{frame} \frametitle{Ende}
	\centering
	\textbf{Fragen?}
\end{frame}

\end{document}

