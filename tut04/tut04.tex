\documentclass{beamer}
\usepackage{../tut-slides}
\usepackage{../mathoperatorsAuD}

\usepackage{csquotes}
\usepackage{cancel}

\usepackage{amsmath,amssymb}
\usepackage{enumerate}
\usepackage[inline]{enumitem} 		%customize label
\newcommand{\labelitemi}{\raisebox{1pt}{\scalebox{.9}{$\blacktriangleright$}}}
\newcommand{\labelitemii}{$\vartriangleright$}
\newcommand{\labelitemiii}{--}

\usepackage{booktabs}
\usepackage{tabularx}
\usepackage{tabu}
\newcommand*\head{\rowfont{\bfseries}}
\newcommand*{\tw}{\rowfont{\ttfamily}}

\renewcommand{\tabularxcolumn}[1]{>{\hspace{0pt}}m{#1}}

\usepackage{listings}
\usepackage[skins]{tcolorbox}
\tcbuselibrary{listings}
\newtcblisting{codebox}{%
	top=0pt,
	bottom=0pt,
	left=0pt,
	right=0pt,
	boxsep=5pt,
	enhanced,clip upper,
	colback=white,
	colframe=cddarkblue,
	fonttitle=\bfseries,
	listing only}
\newtcblisting{smallcodebox}{%
	top=0pt,
	bottom=0pt,
	left=0pt,
	right=0pt,
	boxsep=5pt,
	enhanced,clip upper,
	colback=white,
	colframe=cddarkblue,
	fonttitle=\bfseries,
	fontupper=\tiny,
	listing only}

\newtcblisting{commandshell}{%
	colback=black,colupper=white,colframe=yellow!75!black,
	listing only,listing options={style=tcblatex,language=sh},
	every listing line={\textcolor{red}{\small\ttfamily\bfseries root \$> }}}

\begin{document}	
	\title{Programmierung}
	\subtitle{Übung 3: Bäume \& Funktionen höherer Ordnung}
	\author{Eric Kunze}
	\email{eric.kunze@mailbox.tu-dresden.de}
%	\city{TU Dresden}
	\date{}
%	\institute{Lehrstuhl für Grundlagen der Programmierung}
%	\titlegraphic{\includegraphics[width=2cm]{../TUD-white.pdf}}
	
	\maketitle
	

%%%%%%%%%%%%%%%%%%%%%%%%%%%%%%%%%%%%%%%%%%%%%%%%%%%%%%%%%%%%%%%%%%%%%%%%%%%%%


\section{Übungsblatt 2 --- Aufgabe 3 \\ \textit{\normalsize Algebraische Datentypen}}

\begin{frame}[fragile] \frametitle{Algebraische Datentypen}
	\footnotesize
	\begin{itemize}
		\item Ziel: problemspezifische Datenkonstruktoren
		\item z.B. in \textit{C}: Aufzählungstypen
		\item funktionale Programmierung: algebraische Datentypen
	\end{itemize}
	
	\textbf{Aufbau:}
	\begin{codebox}
data Typename 
	= Con1 t11 ... t1k1
	| Con2 t21 ... t2k2
	| ...
	| Conr tr1 ... trkr
	\end{codebox}
	\begin{itemize}
		\item \texttt{Typename} ist ein Name (Großbuchstabe)
		\item \texttt{Con1, ... Conr} sind Datenkonstruktoren (Großbuchstabe)
		\item \texttt{tij} sind Typnamen (Großbuchstaben)
	\end{itemize}
\end{frame}

\begin{frame}[fragile] \frametitle{Algebraische Datentypen -- Beispiele}
	
	\begin{codebox}
data Typename 
	= Con1 t11 ... t1k1
	| Con2 t21 ... t2k2
	| ...
	| Conr tr1 ... trkr
	\end{codebox}
	
	\begin{codebox}
data Season = Spring | Summer | Autumn | Winter
	\end{codebox}
	\pause
	\begin{codebox}
goSkiing :: Season -> Bool
goSkiing Winter = True
goSkiing _      = False
	\end{codebox}
	\pause
	\begin{codebox}
data Weather = Sunny Int Int Bool | Cloudy Float 
                                  | Rainy String Int 
	\end{codebox}
\end{frame}

\begin{frame}[t, fragile] \frametitle{Aufgabe 3 -- Teil (a)}
	\begin{codebox}
data BinTree = Branch Int BinTree BinTree | Nil
	\end{codebox}
	\pause
	
	Ein Beispielbaum: \\[6pt]
	\begin{codebox}
mytree :: BinTree 
mytree = Branch 0 
         ( Nil )
         ( Branch 3 
            ( Branch 1 Nil Nil )
            ( Branch 5 Nil Nil )
         )
	\end{codebox}
	... erfüllt die Suchbaumeigenschaft.
\end{frame}


\begin{frame}[t, fragile] \frametitle{Aufgabe 3 -- Teil (b)}
	\textbf{Test auf Baum-Gleichheit}
	
	\pause
	
	\begin{codebox}
data BinTree = Branch Int BinTree BinTree | Nil
equal :: BinTree -> BinTree -> Bool
	\end{codebox}
	
	\bigskip \pause
	
	\begin{codebox}
equal :: BinTree -> BinTree -> Bool
equal Nil              Nil              = True
equal Nil              (Branch y l2 r2) = False
equal (Branch x l1 r1) Nil              = False
equal (Branch x l1 r1) (Branch y l2 r2)
	= (x == y) && (equal l1 l2) && (equal r1 r2)
	\end{codebox}
\end{frame}

\begin{frame}[t, fragile] \frametitle{Aufgabe 3 -- Teil (c)}
	\textbf{Einfügen von Schlüsseln in einen Binärbaum}
	
	\begin{codebox}
data BinTree = Branch Int BinTree BinTree | Nil 
insert :: BinTree -> [Int] -> BinTree
	\end{codebox}
	
	\bigskip \pause
	
	\begin{codebox}
insert :: BinTree -> [Int] -> BinTree
insert t     [] = t
insert t (x:xs) = insert t' xs
  where 
    t' = insertSingle t x
    insertSingle Nil            x = Branch x Nil Nil
    insertSingle (Branch y l r) x
  	    | x < y     = Branch y (insertSingle l x) r
  	    | otherwise = Branch y l (insertSingle r x)
	\end{codebox}
\end{frame}

\section{Übungsblatt 3 --- Aufgabe 1 \\ \textit{\normalsize Algebraische Datentypen}}

\begin{frame}[t, fragile] \frametitle{Aufgabe 1 -- Teil (a)}
	\textbf{Anzahl der Blätter}
	
	\begin{codebox}
data RoseTree = Node Int [RoseTree]
countLeaves :: RoseTree -> Int
	\end{codebox}
	
	
	\pause \bigskip
	
	\begin{codebox}
countLeaves :: RoseTree -> Int
countLeaves (Node _ [] )    = 1
countLeaves (Node _ [t])    = countLeaves t
countLeaves (Node x (t:ts)) 
	= countLeaves t + countLeaves (Node x ts)
	\end{codebox}
\end{frame}

\begin{frame}[t, fragile] \frametitle{Aufgabe 1 -- Teil (b)}
	\textbf{gerade Anzahl an Kindern}
	
	\begin{codebox}
data RoseTree = Node Int [RoseTree]
evenNodes :: RoseTree -> Bool
	\end{codebox}
	
	
	\pause \bigskip
	\begin{onlyenv}<2>
		\begin{codebox}
evenNodes :: RoseTree -> Bool
evenNodes (Node _  []       ) = True
evenNodes (Node x  [t]      ) = False
evenNodes (Node x (t1:t2:ts)) 
	= evenNodes (Node x ts) && evenNodes t1 && 
	  evenNodes t2
		\end{codebox}
	\end{onlyenv}

	\begin{onlyenv}<3>
		\begin{codebox}
evenNodes' :: RoseTree -> Bool
evenNodes' (Node _ []) = True
evenNodes' (Node _ ts)
	= mod (length ts) 2 == 0 && evenNodes'' ts
	where
		evenNodes'' :: [RoseTree] -> Bool
		evenNodes'' []     = True
		evenNodes'' (t:ts) 
			= evenNodes' t && evenNodes'' ts
		\end{codebox}
	\end{onlyenv}
\end{frame}


%%%%%%%%%%%%%%%%%%%%%%%%%%%%%%%%%%%%%%%%%%%%%%%%%%%%%%%%%%%%%%%%%%%%%%%%%%%%%%%%%%%%

\section{Übungsblatt 3 --- Aufgaben 2 \& 3 \\ \textit{\normalsize Funktionen höherer Ordnung}}


\begin{frame}[fragile] \frametitle{Funktionen}
	\footnotesize
	Wir kennen bereits einige Möglichkeiten Funktionen zu notieren. Hier seien einige weitere erwähnt.
	\begin{description}
		\item[anonyme Funktionen.] Funktionen ohne konkreten Namen \\
		z.B. \lstinline[basicstyle=\ttfamily\normalsize]|(\x -> x+1)| ist die Addition mit \texttt{1} 
		\begin{codebox}
(\x -> x+1) 4 = 5
		\end{codebox}
		%
		\pause
		%
		\item[Operator $\leftrightarrow$ Funktion] Aus Operatoren (wie z.B. \texttt{+}) kann man eine Funktion machen und vice versa.
		\begin{itemize}
			\item Operator $\to$ Funktion: Klammern
			\begin{codebox}
(+) :: Int -> Int -> Int
(+) x y =  x + y
			\end{codebox}
			\item Funktion $\to$ Operator: Backticks \texttt{`...`}
			\begin{codebox}
5 `mod` 2 = 1
			\end{codebox}
		\end{itemize}
	\end{description}
\end{frame}

\begin{frame}[fragile] \frametitle{Funktionskomposition}
	Analog zur mathematischen Notation $f = g \circ h$ für $f(x) = g(h(x))$ versteht auch Haskell das Kompositionsprinzip mit dem Operator \texttt{.} \\
	z.B. 
	\begin{codebox}
sqAdd :: Int -> Int
sqAdd = (^2) . (+ 5)
	\end{codebox} 
	statt \lstinline[basicstyle=\ttfamily\normalsize]|sqAdd x = (x + 5)^2| für das Quadrat des fünften Nachfolgers
\end{frame}

\begin{frame}[fragile] \frametitle{Partielle Applikation}
	\footnotesize
	Funktionen müssen nicht immer mit allen Argumenten versorgt werden. Lässt man (hintere) Argumente weg, so spricht man von Unterversorgung.
	Die Modulo Funktion hat eigentlich zwei Argumente. Lassen wir das zweite Argument weg, so liefert dies uns eine neue Funktion, die noch ein Argument entgegennimmt und sodann die Restberechnung ausführt.
	\begin{codebox}
mod :: Int -> Int -> Int
mod m n = ...

mod 10 :: Int -> Int
(mod 10) n = mod 10 n
	\end{codebox}
	\pause
	\begin{codebox}
(> 3) :: Int -> Bool
(> 3) x = x > 3
	\end{codebox}
\end{frame}

\begin{frame}[fragile] \frametitle{Funktionen höherer Ordnung --- map}
	\footnotesize
	Funktionen können als Argumente von Funktionen auftreten. Wir lernen drei Basics kennen: \pause
	
	\textbf{Die Funktion map}
	\begin{itemize}
		\item \texttt{map} ermöglicht es eine Funktion \texttt{f} auf alle Elemente einer Liste anzuwenden \\[6pt]
		\begin{codebox}
map :: (Int -> Int) -> [Int] -> [Int] 
map f [] = [] 
map f (x:xs) = f x : map f xs
		\end{codebox}
		\item \textit{Beispiel.} \\[6pt]
		\begin{codebox}
map square [1,2,7,12,3,20] = [1,4,49,144,9,400]
		\end{codebox}
	\end{itemize}
\end{frame}

\begin{frame}[fragile] \frametitle{Funktionen höherer Ordnung --- filter}
	\footnotesize
	\textbf{Die Funktion filter}
	\begin{itemize}
		\item \texttt{filter p xs} liefert eine Liste, die genau die Elemente von \texttt{xs} enthält, welche das Prädikat \texttt{p} erfüllen \\[6pt]
		\begin{codebox}
filter :: (a -> Bool) -> [a] -> [a] 
filter p xs = [ x | x <- xs, p x]
		\end{codebox}
		\item \textit{Beispiel.} \\[6pt]
		\begin{codebox}
filter odd [1,2,7,12,3,20] = [1,7,3]
		\end{codebox}
	\end{itemize}
\end{frame}

\begin{frame}[fragile] \frametitle{Funktionen höherer Ordnung --- foldr}
	\footnotesize
	\textbf{Die Funktion foldr}
	\begin{itemize}
		\item \texttt{foldr f z xs} faltet eine Liste \texttt{xs} und verknüpft jeweils durch die Funktion \texttt{f}; gestartet wird mit \texttt{z} und dem rechtesten Element \\[6pt]
		\begin{codebox}
foldr :: (a -> b -> b) -> b -> [a] -> b
foldr f z []     = z 
foldr f z (x:xs) = f x (foldr f z xs) 
		\end{codebox}
		\item \emph{Beispiel.} \\[6pt]
		\begin{codebox}
foldr (+) 3 [1,2,3,4,5] = 18
length xs = foldr (+) 0 (map (\x -> 1) xs)
		\end{codebox}
	\end{itemize}
\end{frame}


\begin{frame}[fragile] \frametitle{Funktionen höherer Ordnung -- Übersicht}
	\footnotesize
	\begin{itemize}
		\item \texttt{map} wendet Funktion auf alle Listenelemente an \\[6pt]
		\begin{codebox}
map :: (a -> b) -> [a] -> [b]
map f [] = []
map f (x:xs) = f x : map f xs
		\end{codebox}
		\item \texttt{filter} wählt Listenelemente anhand einer Funktion aus \\[6pt]
		\begin{codebox}
filter :: (a -> Bool) -> [a] -> [a]
filter p xs = [ x | x <- xs, p x]
		\end{codebox}
		\item \texttt{foldr} faltet eine Liste mit Verknüpfungsfunktion (von rechts beginnend) \\[6pt]
		\begin{codebox}
foldr :: (a -> b -> b) -> b -> [a] -> b
foldr f z []     = z
foldr f z (x:xs) = f x (foldr f z xs) 
		\end{codebox}
	\end{itemize}
\end{frame}

\begin{frame}[t, fragile] \frametitle{Aufgabe 2}
	\footnotesize
	\textbf{Produkt der Quadrate aller geraden Zahlen einer Liste} \\[6pt]
	\begin{codebox}
f :: [Int] -> Int
	\end{codebox}
	
	\pause \bigskip 
	
	\begin{codebox}
f xs 
  = foldr (+) 0 (map (^2) (filter (`mod` 2) == 0) xs))
	\end{codebox}
	\pause
	\begin{codebox}
f' xs = foldr (*) 1 (map (^2) (filter even xs))
	\end{codebox}
	\pause
	\begin{codebox}
f'' = foldr (*) 1 . map (^2) . filter even
	\end{codebox}
	\begin{codebox}
f''' 
  = foldr (*) 1 . map (^2) . filter ((== 0) . (`mod` 2))
	\end{codebox}
\end{frame}


%%%%%%%%%%%%%%%%%%%%%%%%%%%%%%%%%%%%%%%%%%%%%%%%%%%%%%%%%%%%%%%%%%%%%%%%%%%%%%%%%%%


\begin{frame}[t, fragile] \frametitle{Aufgabe 3}
	\textbf{Faltung einer Liste von \textit{links}} \\[6pt]
	\begin{codebox}
foldleft ::  (Int -> Int -> Int) -> Int -> [Int] -> Int
	\end{codebox}
	
	
	\pause \bigskip
	
	\begin{codebox}
foldleft :: (Int -> Int -> Int) -> Int -> [Int] -> Int
foldleft f x []     = x
foldleft f x (y:ys) = foldleft f (f x y) ys
	\end{codebox}
\end{frame}


\begin{frame} \frametitle{Ende}
	\centering
	\textbf{Fragen?}
\end{frame}

\end{document}

