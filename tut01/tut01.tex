\documentclass{beamer}
\usepackage{../tut-slides}
\usepackage{../mathoperatorsAuD}

\usepackage{lmodern}
\usepackage{amsmath,amssymb}
\usepackage{wasysym}
\usepackage{stmaryrd}
\usepackage{enumerate}
%\usepackage[inline]{enumitem} 		%customize label
%\newcommand{\labelitemi}{\raisebox{1pt}{\scalebox{.9}{$\blacktriangleright$}}}
%\newcommand{\labelitemii}{$\vartriangleright$}
%\newcommand{\labelitemiii}{--}
\setbeamertemplate{itemize item}{\raisebox{1pt}{\scalebox{.9}{$\blacktriangleright$}}}
\setbeamertemplate{itemize subitem}{$\vartriangleright$}



\usepackage{booktabs}
\usepackage{tabularx}
\usepackage{tabu}
\newcommand*\head{\rowfont{\bfseries}}
\newcommand*{\tw}{\rowfont{\ttfamily}}
\renewcommand{\tabularxcolumn}[1]{>{\hspace{0pt}}m{#1}}
\usepackage{multirow}

\usepackage{cancel}

\usepackage{empheq}
\newcommand*\widefbox[1]{\fbox{\hspace{2em} #1 \hspace{2em}}}

\usepackage{xcolor}
\usepackage{tcolorbox}
\newtcolorbox{mymathbox}[1][]{colback=white, sharp corners, #1}

\usepackage{xcolor}
\usepackage{MnSymbol}

\usepackage{tikz}
\usetikzlibrary{patterns,arrows, arrows.meta,calc,decorations.pathmorphing,backgrounds, positioning,fit,automata,shapes,matrix}

\newcommand{\col}[1]{\textcolor{cdpurple}{#1}}
\newcolumntype{R}[1]{>{\centering\arraybackslash}p{#1}}
\usepackage{tabularx}
\renewcommand{\tabularxcolumn}[1]{m{#1}}

\usepackage{ragged2e}
\usepackage{csquotes}

\newcommand{\ghost}[1]{\raisebox{0pt}[0pt][0pt]{\makebox[0pt][l]{#1}}}

%%%% FARBEN %%%%%
\newcommand{\orange}[1]{\textcolor{cdorange}{#1}}
\newcommand{\green}[1]{\textcolor{cdgreen}{#1}}
\newcommand{\purple}[1]{\textcolor{cdpurple}{#1}}
\newcommand{\gray}[1]{\textcolor{cdgray}{#1}}


\usepackage{listings}
\usepackage{tcolorbox}
\tcbuselibrary{listings}

%\lstset{ 
%	basicstyle=\footnotesize\ttfamily,        % the size of the fonts that are used for the code
%	breakatwhitespace=false,         % sets if automatic breaks should only happen at whitespace
%	breaklines=true,                 % sets automatic line breaking
%	commentstyle=\itshape,    	     % comment style
%	escapeinside={\%*}{*)},          % if you want to add LaTeX within your code
%	extendedchars=true,              % lets you use non-ASCII characters; for 8-bits encodings only, does not work with UTF-8
%	firstnumber=1,                % start line enumeration with line 1000
%	frame=single,
%	keywordstyle=\bfseries,       % keyword style
%	language=Haskell,                 % the language of the code
%	numbers=left,                    % where to put the line-numbers; possible: (none, left, right)
%	numbersep=5pt,                   % how far the line-numbers are from the code
%	numberstyle=\tiny\color{cdgray!50}, % the style that is used for the line-numbers
%	rulecolor=\color{cddarkblue}, 
%	tabsize=2,	                   % sets default tabsize to 2 spaces
%}

\lstdefinestyle{bg}{ 
	basicstyle=\footnotesize\ttfamily,        % the size of the fonts that are used for the code
	breakatwhitespace=false,         % sets if automatic breaks should only happen at whitespace
	breaklines=true,                 % sets automatic line breaking
	commentstyle=\itshape,    	     % comment style
	escapeinside={\%*}{*)},          % if you want to add LaTeX within your code
	extendedchars=true,              % lets you use non-ASCII characters; for 8-bits encodings only, does not work with UTF-8
	firstnumber=1,                % start line enumeration with line 1000
	frame=L,
	backgroundcolor=\color{cdgray!10},
%	keywordstyle=\bfseries,       % keyword style
	language=Haskell,                 % the language of the code
	numbers=left,                    % where to put the line-numbers; possible: (none, left, right)
	numbersep=8pt,                   % how far the line-numbers are from the code
	numberstyle=\tiny\color{cdgray!50}, % the style that is used for the line-numbers
	rulecolor=\color{cddarkblue}, 
	tabsize=2,	                   % sets default tabsize to 2 spaces
}

\newtcblisting{codebox}{%
	colback=white,
	colframe=cddarkblue,
	fonttitle=\bfseries,
	listing only}




\begin{document}	
	\title{Programmierung}
	\subtitle{Übung 1: Einleitung}
	\author{Eric Kunze}
	\email{eric.kunze@tu-dresden.de}
	\city{TU Dresden}
	\date{13. April 2022}
%	\institute{Lehrstuhl für Grundlagen der Programmierung}
	\titlegraphic{\includegraphics[width=2cm]{../TUD-white.pdf}}

	\maketitle
	
	\begin{frame} \frametitle{Wer bin ich?}
		\footnotesize
		\begin{minipage}{\dimexpr0.75\linewidth-\fboxrule-\fboxsep}
			\flushleft \normalsize
			\begin{itemize}
				\item \textbf{Eric} \textcolor{cdgray!50}{Kunze}
				\item \url{eric.kunze@tu-dresden.de}
				\item Fragen, Wünsche, Vorschläge, \dots 
				\item \textbf{Telegram:} \\ 
				\texttt{@oakoneric} bzw. \url{t.me/oakoneric}
			\end{itemize}
		\end{minipage}
		\begin{minipage}{\dimexpr0.25\linewidth-\fboxrule-\fboxsep}
			\includegraphics[width=3cm]{./tut01_pic.jpg}
		\end{minipage}		
	
		Meine Materialien sind auf meiner Website \url{https://oakoneric.github.io/prog22} zu finden.
		\begin{itemize}
			\item Slides (mit Lösungen); evtl. zusätzliche Materialien (nach Bedarf)
			\item \alert{\textbf{kein Anspruch auf Vollständigkeit \& Korrektheit}}
		\end{itemize}
		
		Source Code auf Github \\
		\url{https://github.com/oakoneric-tutorials/programmierung-ss22}
	\end{frame}
	
	\begin{frame}\frametitle{Infos \& Co}		
		\textbf{OPAL-Kurs}: 
		\url{https://tud.link/y47l}
		
		\begin{itemize}
			\item alle Informationen zur Lehrveranstaltung
			\item Link zur Vorlesung: Freitag, 2. DS via Zoom
			\item Übungsblätter
			\item Forum
		\end{itemize}
		
		\textbf{Materialien}: 
		\begin{itemize}
			\item Slides der Vorlesung (ersetzen ehemaliges Skript)
			\item Aufgabensammlung
		\end{itemize}
		
		
		\textbf{Verlegung am Ostermontag}: \\
		Ersatztermin am \textit{Mittwoch, 20.04.2022 um 13 Uhr}
		\begin{itemize}
			\item Information per Mail zu Raum/online
		\end{itemize}
	\end{frame}
	
	
%	\begin{frame} \frametitle{Was wird in der Übung erwartet?}
%		\textbf{Ziel:} viel Interaktivität
%		
%		\textbf{Mein Input}
%		\begin{itemize}
%			\item Zusammenfassung einiger Vorlesungsinhalte
%			\item beispielhafte Lösungsansätze und Lösungen
%			\item Fragen, Fragen, Fragen
%		\end{itemize}
%		
%		\pause
%		
%		\textbf{Euer Input}
%		\begin{itemize}
%			\item Grundverständnis aus der Vorlesung
%			\item Vorbereitung der Übungsaufgaben
%			\item aktive Mitarbeit und Fragen
%		\end{itemize}
%	\end{frame}

	\begin{frame} \frametitle{Hinweise}
		\textbf{Literatur}
		\begin{itemize}
			\item \textit{Learn You a Haskell For Great Good!}
			\begin{itemize}
				\item sehr gut geschrieben, ausführlich und kurzweilig
			\end{itemize}
			\item \textit{Real World Haskell}
			\begin{itemize}
				\item sehr gut, wesentlich mehr Inhalte als benötigt
			\end{itemize}
		\end{itemize}
		beide Bücher als Online-Versionen verfügbar (siehe Website)
		
		\medskip \pause
		
		\textbf{Altklausuren}
		\begin{itemize}
			\item FTP-Server des iFSR:
			\url{https://ftp.ifsr.de/klausuren/Grundstudium/Programmierung/}
			\item VPN-Verbindung notwendig
		\end{itemize}
	\end{frame}
%%%%%%%%%%%%%%%%%%%%%%%%%%%%%%%%%%%%%%%%%%%%%%%%%%%%%%%%%%%%%%%%%%%%%%%%%%%%%

\section{Einführung in Haskell}

    \begin{frame}[fragile] \frametitle{Haskell \& funktionale Programmierung}
    	\footnotesize
        \fbox{Haskell $=$ funktionale Programmiersprache}
        
        Wir programmieren nicht \textit{wie} berechnet wird, sondern \textit{was} berechnet wird.
    	
    	\pause
        \medskip
        
        \textbf{Mathe:} Wir kennen Funktionen bereits aus dem Mathe-Unterricht und den Mathe-Vorlesungen. Zum Beispiel ist 
        \begin{equation*}
        	\begin{aligned}
        		f : \N &\to \N \\
        		f(x) &= x + 3
        	\end{aligned}
        \end{equation*}
    	eine Funktion, die natürliche Zahlen auf natürliche Zahlen abbildet.
    	
    	\pause
    	
    	\textbf{Haskell:} 
    	Diese würde in Haskell wie folgt aussehen:
        \begin{lstlisting}[style=bg]
f :: Int -> Int
f x = x + 3
        \end{lstlisting}
    \end{frame}

	\begin{frame}[fragile] \frametitle{Ein weiteres Beispiel}
		\footnotesize
		Um zu verdeutlichen, wie ähnlich sich mathematische Funktionen und Haskell-Funktionen sind, betrachten wir folgendes Beispiel. 
		Wir können Funktionen auf ihren Argumenten definieren, d.h.
		\begin{equation*}
			\begin{aligned}
				g : \N &\to \N \\
				g(0) &= 1 \\
				g(x) &= x^2
			\end{aligned}
		\end{equation*}
		bzw. in Haskell
		\begin{lstlisting}[style=bg]
g :: Int -> Int
g 0 = 1
g x = x * x
		\end{lstlisting}
	\end{frame}

	\begin{frame}[fragile] \frametitle{Ein weiteres Beispiel}
		\footnotesize
		Wir können auch deutlich wichtigere und kompliziertere Funktionen programmieren. Zum Beispiel lässt sich die Addition $n+m$ auch als Funktion schreiben.
		\begin{equation*}
			\operatorname{add} : \mathbb{N} \times \mathbb{N} \to \mathbb{N}, \quad
			\operatorname{add}(n,m) = n + m = 
			\begin{cases}
				n & m = 0 \\
				1 + \operatorname{add} ( n , m-1 ) &\text{sonst} \\
			\end{cases}
		\end{equation*}
		definieren.
		
		\medskip \pause
		
		Als Haskell-Funktion sieht das dann so aus:
		\begin{lstlisting}[style=bg]
add :: Int -> Int -> Int
add n 0 = n
add n m = 1 + add n (m-1)
		\end{lstlisting}
	\end{frame}
	
	
	\section{Aufgabe 1 \\ \textit{\normalsize Haskell installieren und compilieren}}

	\begin{frame}[fragile] \frametitle{Aufgabe 1}
		Gegeben sei eine Haskell-Funktion
		\begin{lstlisting}[style=bg]
sum3 :: Int -> Int -> Int -> Int
sum3 x y z = x + y + z
		\end{lstlisting}
	
		\pause
		
		Die entspricht der mathematische Funktion
		\begin{equation*}
			\operatorname{sum3} \colon \N \times \N \times \N \to \N \quad \text{mit} \quad \operatorname{sum3}(x,y,z) = x + y + z
		\end{equation*}
	\end{frame}

	\begin{frame}\frametitle{Haskell installieren und compilieren}
		\textbf{Glasgow Haskell Compiler} (ghc(i)) :  \url{https://www.haskell.org/ghc/}
		
		\pause
		
		\begin{itemize}
			\item \textbf{Terminal:} \texttt{ghci <modulname>}
			\item \textbf{Module laden: } \texttt{:load <modulname>} oder \texttt{:l}
			\item \textbf{Module neu laden: } \texttt{:reload} oder \texttt{:r}
			\item \textbf{Hilfe: } \texttt{:?} oder \texttt{:help}
			\item \textbf{Interpreter verlassen: } \texttt{:quit} oder \texttt{:q}
			
			\bigskip \pause
			
			\item \texttt{:type <exp>} --- Typ des Ausdrucks \texttt{<exp>} bestimmen
			\item \texttt{:info <fkt>} --- kurze Dokumentation für \texttt{<fkt>}
			\item \texttt{:browse} --- alle geladenen Funktionen anzeigen
			
			\bigskip \pause
			
			\item einzeilige Kommentare mit \texttt{--}
			\item mehrzeilige Kommentare mit \texttt{\{- ... -\}}
		\end{itemize}
	\end{frame}



	\section{Aufgabe 2 \\ \textit{\normalsize Rekursion, Pattern Matching \& Conditionals}}
	
	\begin{frame}\frametitle{Das Prinzip der Rekursion}
		\footnotesize
		Ein wichtiges Prinzip in der funktionalen Programmierung ist das Prinzip der Rekursion.
		
		\begin{itemize}
			\item \textbf{Rekursionsfall}
			\onslide<2->{\begin{itemize} \footnotesize
				\item Reduktion eines großen Prolems auf ein kleineres Problem
				\item \texttt{Int}-Funktionen: Reduktion von $n$ auf $n-1$
				\item Liste: Reduktion durch Abspaltung eines Listenelements
			\end{itemize}}
		
			\bigskip
			
			\item \textbf{Basisfall}
			\onslide<3->{\begin{itemize} \footnotesize
				\item kleinste Probleme - einfach zu lösen
				\item rechte Seite deterministisch
				\item \texttt{Int}-Funktionen: rechte Seite hängt nicht von $n$ ab
				\item Liste: leere Liste
			\end{itemize}}
		\end{itemize}
	\end{frame}
	
	\begin{frame}[fragile] \frametitle{Pattern Matching}
		\footnotesize
		Mit Pattern Matching kann man prüfen, ob Funktionsargumente eine bestimmte Form aufweisen.
		
		Damit kann man verschiedene Fälle in einfacher Form nacheinander abgreifen, z.B. Basis- und Rekursionsfall. Vergleiche dazu auch das Beispiel mit der \texttt{add}-Funktion:
		\begin{itemize} \footnotesize
			\item Der Aufruf \texttt{add 5 0} matched mit Zeile 2, also berechnen wir \texttt{add 5 0 = 5}.
			\item Der Aufruf \texttt{add 5 1} matched nicht auf Zeile 2, also probieren wir Zeile 3. Das matched mit \texttt{n = 5} und \texttt{m = 1} und wir berechnen
			\begin{align*}
				\texttt{add} \ 5 \ 1 &= 1 + \texttt{add} \ 5 \ 0 \\
				        &= 1 + 5 \\
				        &= 6
			\end{align*}
		\end{itemize}
		Beachte, dass dabei von oben nach unten getestet wird!
	\end{frame}


	\begin{frame}[fragile] \frametitle{Conditionals}
		\footnotesize
		Um Bedingungen zu testen, gibt es die Möglichkeit auf \texttt{if}-\texttt{then}-\texttt{else} zu verzichten und sogenannte \textit{guards} mit \textit{pipes} zu verwenden. Das sieht dann wieder so aus, wie eine geschweifte Klammer in mathematischen Fallunterscheidungen. 
		\begin{equation*}
			h(x) = \begin{cases}
				x^2 & \text{für } x < 0 \\
				0.5*x & \text{für } \onslide*<1>{x \ge 0} \onslide*<2>{\cancel{x \ge 0} \text{ \textbf{sonst}}}
			\end{cases} 
		\end{equation*}
		\begin{onlyenv}<1>
			\begin{lstlisting}[style=bg]
h :: Int -> Int
h x 
	| x < 0  = x^2 
	| x >= 0 = 0.5 * x
			\end{lstlisting}
		\end{onlyenv}
		\begin{onlyenv}<2>
			\begin{lstlisting}[style=bg]
h :: Int -> Int
h x 
	| x < 0     = x^2 
	| otherwise = 0.5 * x
			\end{lstlisting}
		\end{onlyenv}
		\pause
		Wie auch in Mathe sollte man bei gegensätzlichen Bedingungen ein \enquote{sonst} bzw. \texttt{otherwise} verwenden.
	\end{frame}


	\begin{frame}\frametitle{Aufgabe 2a -- Fakultätsfunktion}
		\footnotesize
		Fakultät
		\begin{equation*}
			n! = \prod_{i=1}^n i
		\end{equation*}
		
		\medskip
		\pause
		
		Rekursionsvorschrift: $n \leadsto n-1$ \pause
		\begin{equation*}
			n! = \boldsymbol{n} * \prod_{i=1}^{\boldsymbol{n-1}} i= n * (n-1)!
		\end{equation*}
		\begin{itemize}
			\item links: $n!$ hängt von $n$ ab
			\item rechts: $(n-1)!$ hängt nur von $n-1$ ab
		\end{itemize}
		
		\medskip
		\pause
		
		Um die Rekursion vollständig zu definieren, benötigen wir einen \textit{Basisfall}. Wann können wir also die Rekursion der Fakultät abbrechen?
		\begin{equation*}
			0 ! = 1 \qquad 1 ! = 1 \qquad 2! = 2 \qquad \cdots
		\end{equation*}
		$\Rightarrow$ Welcher Basisfall ist sinnvoll? \qquad $0! = 1$
	\end{frame}

	\begin{frame}[fragile] \frametitle{Aufgabe 2a -- Lösung}
		\begin{lstlisting}[style=bg]
fac :: Int -> Int
fac 0 = 1
fac n = n * fac (n-1)
		\end{lstlisting}
		
		\scriptsize
		\textbf{Hinweis:} In der Musterlösung werden noch \texttt{undefined}-Fälle angegeben. Das ist für uns erst einmal optional, aber natürlich schöner.
	\end{frame}


	\begin{frame}[fragile] \frametitle{Aufgabe 2b --- Summierte Fakultäten (1)}
		\footnotesize
		\begin{equation*}
			f(n,m) = \sum_{i=n}^{m} i!
		\end{equation*}
		\pause
		\begin{itemize}
			\item Rekursionsfall: 
			\begin{equation*}
				f(n,m) = \sum_{i=n}^{m} i! = \boldsymbol{m}! + \sum_{i=n}^{\boldsymbol{m-1}} i! = m! + f(n,m-1) 
			\end{equation*}
			\item Basisfall: $f(n,m) = 0$ für $n > m$
		\end{itemize}
		\pause
		
		\textbf{Lösung:} \\[0.5em]
		\begin{lstlisting}[style=bg]
sumFacs :: Int -> Int -> Int
sumFacs n m
	| n > m = 0
	| otherwise = fac m + sumFacs n (m-1)
		\end{lstlisting}
	\end{frame}

	\begin{frame}[fragile] \frametitle{Aufgabe 2b --- Summierte Fakultäten (2)}
		\footnotesize
		\begin{equation*}
			f(n,m) = \sum_{i=n}^{m} i!
		\end{equation*}
		\pause
		\begin{itemize}
			\item Rekursionsfall: 
			\begin{equation*}
				f(n,m) = \sum_{i=n}^{m} i! = \boldsymbol{n}! + \sum_{i=\boldsymbol{n+1}}^{m} i! = n! + f(n+1,m) 
			\end{equation*}
			\item Basisfall: $f(n,m) = 0$ für $n > m$
		\end{itemize}
		\pause
		
		\textbf{Lösung:} \\[0.5em]
		\begin{lstlisting}[style=bg]
sumFacs :: Int -> Int -> Int
sumFacs n m
	| n > m = 0
	| otherwise = fac n + sumFacs (n+1) m
		\end{lstlisting}
	\end{frame}


	\section{Aufgabe 3}
	
	\begin{frame}\frametitle{Aufgabe 3 -- Fibonacci-Zahlen}
		\scriptsize
		\begin{equation*}
		f_n := \begin{cases}
		\enskip 1 & \text{falls } n = 0 \\
		\enskip 1 & \text{falls } n = 1 \\
		\enskip f_{n-1} + f_{n-2} & \text{sonst}
		\end{cases}
		\end{equation*}
		
		\medskip
		\pause
		
		$\Rightarrow$ Rekursionsvorschrift schon gegeben. \\
		
		\bigskip 
		\begin{minipage}{\dimexpr0.5\linewidth-\fboxrule-\fboxsep}
			\textbf{Verfahren ohne Rekursion.} \\
			
			\begin{tikzpicture}[]%
			\pgfmathsetseed{1237}%
			\node (n1) [draw=cdindigo, fill=cdindigo!10,thick,align=justify, , inner ysep=2mm, inner xsep=2mm, rectangle] at (0,0) {$f_{i-2}$};%
			\node (n2) [draw=cdindigo, fill=cdindigo!10,thick,align=justify, , inner ysep=2mm, inner xsep=2mm, rectangle] at (1.5,0) {$f_{i-1}$};%
			\node (n3) [draw=cdindigo, fill=cdindigo!10,thick,align=justify, , inner ysep=2mm, inner xsep=2mm, rectangle] at (3,0) {$n$};%
			\node (n4) [draw=cdindigo, fill=cdindigo!10,thick,align=justify, , inner ysep=2mm, inner xsep=2mm, rectangle] at (0,-1) {$f_{i-1}$};%
			\node (n5) [draw=cdindigo, fill=cdindigo!10,thick,align=justify, , inner ysep=2mm, inner xsep=2mm, rectangle] at (1.5,-1) {$f_{i-2} + f_{i-1}$};%
			\node (n6) [draw=cdindigo, fill=cdindigo!10,thick,align=justify, , inner ysep=2mm, inner xsep=2mm, rectangle] at (3,-1) {$n-1$};%
			\path[->] (n2) edge [bend left=0] (n4);
			\end{tikzpicture}%
		\end{minipage}
		\pause
		\begin{minipage}{\dimexpr0.5\linewidth-\fboxrule-\fboxsep}
			\textbf{Explizite Formel.}
			\begin{align*}
				f_n = \frac{\Phi^n - \left(-\frac{1}{\Phi} \right)^n}{\sqrt{5}}
				\text{mit} \Phi &= \frac{1 + \sqrt{5}}{2}
			\end{align*}
		\end{minipage}
	\end{frame}
	
	\begin{frame}[fragile] \frametitle{Aufgabe 3 -- Lösung}
		\begin{lstlisting}[style=bg]
fib :: Int -> Int
fib 0 = 1
fib 1 = 1
fib n = fib (n-1) + fib (n-2)
		\end{lstlisting}
	
		\begin{lstlisting}[style=bg]
fib' :: Int -> Int
fib' n = fib_help 1 1 n

fib_help :: Int -> Int -> Int
fib_help x _ 0 = x
fib_help x y n = fib_help y (x+y) (n-1)
		\end{lstlisting}
	\end{frame}



%	%
%	\begin{frame}[fragile] \frametitle{Haskell \& Listen}
%		\begin{description}
%			\item[Listen] Wenn \texttt{a} ein Typ ist, dann bezeichnet \texttt{[a]} den Typ ``Liste mit Elementen vom Typ \texttt{a}'', insbesondere
%			haben alle Elemente einer Liste den gleichen Typ
%			\pause
%			\item[cons-Operator `` \texttt{:} '']  Trennung von head und tail einer Liste \\
%			\lstinline{[ x1 , x2 , x3 , x4 , x5] = x1 : [x2 , x3 , x4 , x5]}
%			\bigskip \pause
%			\item[Verkettungsoperator `` \texttt{++} ''] Verkettung zweier Listen gleichen Typs \\
%			\lstinline{[x1 , x2 ] ++ [x3 , x4 , x5] = [x1 , x2 , x3 , x4 , x5]}
%		\end{description}
%	\end{frame}
%
%	\begin{frame}[fragile] \frametitle{Zeichen \& Zeichenketten}
%		\textbf{Zeichen}
%		\begin{itemize}
%			\item Datentyp \texttt{Char}
%			\item Eingabe in einfachen Anführungszeichen
%			\item z.B. \texttt{'a'}, \texttt{'e'}, \texttt{'3'}
%		\end{itemize}
%	
%		\textbf{Zeichenketten}
%		\begin{itemize}
%			\item Datentyp \texttt{String = [Char]}
%			\item Eingabe in doppelten Anführungszeichen
%			\item z.B. \texttt{"hallo"}, \texttt{"welt"}
%			\item Konkatenation von Zeichenketten: 
%			\begin{lstlisting}[basicstyle=\ttfamily, numbers=none, frame=none, showstringspaces=false]
%"hallo " ++ "welt" = "hallo welt"
%			\end{lstlisting}
%		\end{itemize}
%	\end{frame}
%
%
%\section{Übungsblatt 1}
%

%

\end{document}