\documentclass[ngerman,a4paper, 12pt, fleqn]{article}
\usepackage[top=2cm,bottom=2.5cm,left=2.5cm,right=2.5cm]{geometry}

\usepackage[ngerman]{babel}
\usepackage[utf8]{inputenc}

\usepackage{parskip}
\usepackage{chngcntr}


\usepackage{fancyhdr} 	% customize header / footer
\usepackage{titlesec} 	% customize titles


\usepackage[scale=1]{opensans}
\newcommand*{\osfamily}{\fontfamily{fos}\selectfont}
\DeclareTextFontCommand{\textos}{\osfamily}

%%%%%%%%%%%%%%%%%%%%%%%%%%%%%%%%%%%%%%%%%%%%%%%%%
% math related packages
% basic ams-math and enhancments
\usepackage{amsmath,amssymb,amsfonts,mathtools}
% add some font-related stuff
\usepackage{lmodern}



%%%%%%%%%%%%%%%%%%%%%%%%%%%%%%%%%%%%%%%%%%%%%%%%%
% graphics-related packages
\usepackage[table,dvipsnames]{tudscrcolor} 


%%%%%%%%%%%%%%%%%%%%%%%%%%%%%%%%%%%%%%%%%%%%%%%%%
% text-related packages
% increase line spacing
\usepackage[onehalfspacing]{setspace} % increase row-space

%%%%%%%%%%%%%%%%%%%%%%%%%%%%%%%%%%%%%%%%%%%%%%%%%%%%%%%%%%%%%%%%%%%
%                           REFERENCES                            %
\usepackage[
	type={CC},
	modifier={by-nc-sa},
	version={4.0},
]{doclicense}

\usepackage[unicode,bookmarks=true]{hyperref}
\hypersetup{
	pdfborder={0 0 0},			% no boxed around links
	colorlinks=true,
	linkcolor=cdblue,
	urlcolor=cdblue
%	pdfborderstyle={/S/U/W 1},	% underlining insteas of boxes
%	linkbordercolor=cdblue,
%	urlbordercolor=cdblue
}

\usepackage{cleveref}
\usepackage{bookmark}		% pdf-bookmarks

%%%%%%%%%%%%%%%%%%%%%%%%%%%%%%%%%%%%%%%%%%%%%%%%%%%%%%%%%%%%%%%%%%%%%%%
\DeclareMathSymbol{*}{\mathbin}{symbols}{"01}

\titleformat{\section}{\normalfont\sffamily\LARGE\bfseries}{\thesection}{1em}{}
\titleformat{\subsection}{\normalfont\sffamily\large\bfseries}{\thesection}{1em}{}
\titlespacing*{\subsection}{0pt}{-0.3\baselineskip}{-.8\baselineskip}

\newcommand*\ruleline[1]{\par\noindent\raisebox{.8ex}{\makebox[\linewidth]{\hrulefill\hspace{1ex}\raisebox{-.8ex}{#1}\hspace{1ex}\hrulefill}}}

\newcommand{\menge}[1]{\left\{ #1 \right\}}
\newcommand{\defeq}{\coloneqq}

\newcommand{\befehl}[1]{\textsc{#1}}

\begin{document}
	\hrule
	\vspace{-0.5\baselineskip}
	\section*{Übersetzung $\mathsf{C_0 \to AM_0}$}	
	\hrule
	\begin{center}
		\small \slshape Keine Garantie auf Vollständigkeit und/oder Korrektheit!
	\end{center}
	\bigskip
	
	\subsection*{Rahmenwerk}
	\begin{align*}
		&trans (\texttt{ \#\. include <stdio.h> int main() } block ) \defeq blocktrans(block) \\
		%
		&blocktrans(\texttt{\{} decl \ statseq \ \texttt{return 0;}\texttt{\}}) \defeq stseqtrans(statseq, update(decl , tab_\emptyset ), 1) \\
		%
		&stseqtrans(stat_1 stat_2 \dots stat_n , tab, a) \defeq \\[-3pt]
		&\hspace{5em}\begin{aligned}
			&sttrans(stat_1 , tab, a.1) \\[-3pt]
			&sttrans(stat_2 , tab, a.2) \\[-3pt]
			&... \\[-3pt]
			&sttrans(stat_n , tab, a.n)
		\end{aligned} \\
		%
		&update( \texttt{const } id_1 \texttt{=} z_1 , \dots , id_n \texttt{=} z_n \texttt{; } \texttt{int} id'_1 , \dots , id'_m \texttt{;} , tab) \defeq \\[-3pt]
		&\hspace{5em}
		tab[id_1 / (\operatorname{const}, z_1 ), \dots , id_n /(\operatorname{const}, z_n), id'_1 /(\operatorname{var}, 1), \dots , id'_m /(\operatorname{var}, m)]	
	\end{align*}

	\subsection*{Sequenzen \& Zuweisungen}
	
	\begin{align*}		%
		&sttrans(\texttt{\{} stat_1 stat_2 \dots stat_n \texttt{\}}, tab, a) \defeq stseqtrans(stat_1 stat_2 \dots stat_n , tab, a) \\
		%
		&sttrans(id \texttt{ = } exp\texttt{;}, tab, a) \defeq \\[-3pt]
		&\hspace{5em} \text{if } tab(id) = (\operatorname{var}, n) \text{ then } simpleexptrans(exp, tab) \befehl{ STORE } n\texttt{;}
	\end{align*}

	\subsection*{Input/Output}
	\begin{align*}
		&sttrans(\texttt{scanf("\%d", \&}id\texttt{);}, tab, a) \defeq \text{ if } tab(id) = (\operatorname{var}, n) \text{ then } \befehl{READ } n\texttt{;} \\
		%
		&sttrans(\texttt{printf("\%d", } id\texttt{);}, tab, a) \defeq \text{ if } tab(id) = (\operatorname{var}, n) \text{ then } \befehl{WRITE } n\texttt{;}
	\end{align*}

 	\subsection*{Einfache Expressions}
	
	\begin{align*}
		&boolexptrans(se_1 \ rel \ se_2, tab) \defeq \\[-3pt]
		&\hspace{5em}\begin{aligned}
			&simpleexptrans(se_1 , tab) \\[-3pt]
			&simpleexptrans(se_2 , tab) \\[-3pt]
			&\befehl{REL}\texttt{;} \\[-3pt]
			&\text{wobei } (rel , \befehl{REL}) \in \{ (\texttt{==}, \befehl{EQ}), (\texttt{!=}, \befehl{NE}), (\texttt{<}, \befehl{LT}), (\texttt{>}, \befehl{GT}), (\texttt{<=}, \befehl{LE}), (\texttt{>=}, \befehl{GE}) \} 
		\end{aligned} \\	
		%
		&simpleexptrans(\texttt{x + a * 2}, [\texttt{a}/(\texttt{const}, 5), \texttt{x}/(\texttt{var}, 1)])) = \\[-3pt]
		& \hspace{5em}
		\befehl{LOAD } 1\texttt{; } \befehl{LIT } 5 \texttt{; } \befehl{LIT } 2 \texttt{;} \befehl{MUL} \texttt{; } \befehl{ADD} \texttt{;}	
	\end{align*}

	\subsection*{Verzweigungen}
	\begin{align*}
		&sttrans(\texttt{if ( } exp \texttt{ ) } stat, tab, a) \defeq \\[-3pt]
		& \hspace{5em}
		\begin{aligned}
			&boolexptrans(exp, tab) \\[-3pt]
			&\befehl{JMC } a.1\texttt{;} \\[-3pt]
			&sttrans(stat, tab, a.2) \\[-3pt]
			a.1: \ &
		\end{aligned} \\
		%
		&sttrans(\texttt{if ( } exp \texttt{ ) } stat_1 \texttt{ else } stat_2, tab, a) \defeq \\[-3pt]
		&\hspace{5em}\begin{aligned}
			&boolexptrans(exp, tab) \\[-3pt]
			&\befehl{JMC } a.1\texttt{;} \\[-3pt]
			&sttrans(stat_1, tab, a.2) \\[-3pt]
			&\befehl{JMP } a.3\texttt{;} \\[-3pt]
			a.1: \ &sttrans(stat_2 , tab, a.4) \\[-3pt]
			a.3: \ &
		\end{aligned} 
	\end{align*}

 	\subsection*{Schleifen}
 	
 	\begin{align*}
 		&sttrans(\texttt{while ( } exp \texttt{ ) } stat, tab, a) \defeq \\[-3pt]
 		&\hspace{5em}\begin{aligned}
 			a.1: \ &boolexptrans(exp, tab) \\[-3pt]
 			&\befehl{JMC } a.2 \texttt{;} \\[-3pt]
 			&sttrans(stat, tab, a.3) \\[-3pt]
 			&\befehl{JMP } a.1\texttt{;} \\[-3pt]
 			a.2: \ &
 		\end{aligned}
 	\end{align*}
 

\end{document}