\documentclass{beamer}
\usepackage{../tut-slides}
\usepackage{../mathoperatorsAuD}

\usepackage{csquotes}
\usepackage{cancel}

\usepackage{amsmath,amssymb}
\usepackage{enumerate}
\usepackage[inline]{enumitem} 		%customize label
\newcommand{\labelitemi}{\raisebox{1pt}{\scalebox{.9}{$\blacktriangleright$}}}
\newcommand{\labelitemii}{$\vartriangleright$}
\newcommand{\labelitemiii}{--}

\usepackage{booktabs}
\usepackage{tabularx}
\usepackage{tabu}
\newcommand*\head{\rowfont{\bfseries}}
\newcommand*{\tw}{\rowfont{\ttfamily}}

\renewcommand{\tabularxcolumn}[1]{>{\hspace{0pt}}m{#1}}

\usepackage[skins]{tcolorbox}
\tcbuselibrary{listings}
\newtcblisting{codebox}{%
	top=0pt,
	bottom=0pt,
	left=0pt,
	right=0pt,
	boxsep=5pt,
	enhanced,clip upper,
	colback=white,
	colframe=cddarkblue,
	fonttitle=\bfseries,
	listing only}
\newtcblisting{smallcodebox}{%
	top=0pt,
	bottom=0pt,
	left=0pt,
	right=0pt,
	boxsep=5pt,
	enhanced,clip upper,
	colback=white,
	colframe=cddarkblue,
	fonttitle=\bfseries,
	fontupper=\tiny,
	listing only}

\lstdefinestyle{bg}{ 
	basicstyle=\footnotesize\ttfamily,        % the size of the fonts that are used for the code
	breakatwhitespace=false,         % sets if automatic breaks should only happen at whitespace
	breaklines=true,                 % sets automatic line breaking
	commentstyle=\itshape,    	     % comment style
	escapeinside={\%*}{*)},          % if you want to add LaTeX within your code
	extendedchars=true,              % lets you use non-ASCII characters; for 8-bits encodings only, does not work with UTF-8
	firstnumber=1,                % start line enumeration with line 1000
	frame=L,
	backgroundcolor=\color{cdgray!10},
	%	keywordstyle=\bfseries,       % keyword style
	language=Haskell,                 % the language of the code
	numbers=left,                    % where to put the line-numbers; possible: (none, left, right)
	numbersep=8pt,                   % how far the line-numbers are from the code
	numberstyle=\tiny\color{cdgray!50}, % the style that is used for the line-numbers
	rulecolor=\color{cddarkblue}, 
	tabsize=2,	                   % sets default tabsize to 2 spaces
}
\begin{document}	
	\title{Programmierung}
	\subtitle{Übung 2: Listen, Zeichenketten \& Bäume}
	\author{Eric Kunze}
	\email{eric.kunze@mailbox.tu-dresden.de}
%	\city{TU Dresden}
	\date{}
%	\institute{Lehrstuhl für Grundlagen der Programmierung}
%	\titlegraphic{\includegraphics[width=2cm]{../TUD-white.pdf}}
	
	\maketitle


%%%%%%%%%%%%%%%%%%%%%%%%%%%%%%%%%%%%%%%%%%%%%%%%%%%%%%%%%%%%%%%%%%%%%%%%%%%%%

\section{Übungsblatt 1 \\ \textit{\normalsize Zusatzaufgabe}}

\begin{frame} \frametitle{Übungsblatt 1 -- Zusatzaufgabe}
	\footnotesize
	\textbf{Ziel:} Anzahl der vollständigen Binärbäume mit $n$ Knoten
	
	\textbf{Idee:} Wie erhalten wir volle Binärbäume? --- Ein voller Binärbaum ist
	\begin{itemize}
		\item entweder ein Blatt
		\item oder er besteht aus einer Wurzel und \textit{zwei} Kindern 
	\end{itemize}

	\textbf{Umsetzung:}  
	\begin{itemize}
		\item Rekursionsfall: $n \ge 3$ Knoten 
		\begin{itemize}
			\item ein Wurzelknoten
			\item $n-1$ Knoten für linken und rechten Teilbaum \\ (systematisch alle Möglichkeiten durchlaufen)
		\end{itemize}
		\item Basisfall: 
		\begin{itemize}
			\item $n = 0$: es gibt keinen Baum mit keinen Knoten
			\item $n = 1$: Baum mit einem Knoten $=$ Blatt (davon gibt es genau einen)
		\end{itemize}
	\end{itemize}
\end{frame}

\begin{frame}[fragile] \frametitle{Übungsblatt 1 -- Zusatzaufgabe}
	\footnotesize
	\begin{lstlisting}[style=bg]
countBinTrees :: Int -> Int
countBinTrees 0 = 0
countBinTrees 1 = 1
countBinTrees n = go (n-1)
  where
    go 0 = 0
    go m = go (m-1) + countBinTrees (n - 1 - m) * 
                      countBinTrees m
	\end{lstlisting}

	\textbf{Hinweis:} \texttt{go} durchläuft alle Möglichkeiten $n-1$ Knoten so auf zwei (Kind-)Bäume zu verteilen, dass der linke Teilbaum $m$ Knoten und der rechte Teilbaum die übrigen $n - 1 - m$ Knoten besitzt.
\end{frame}


\section{Aufgabe 1 \\ \textit{\normalsize Binomialkoeffizient}}

\begin{frame}[t, fragile] \frametitle{Aufgabe 1}
	\small
	\textbf{Binomialkoeffizient}
	
	\begin{equation*}
		\binom{n}{k} = \frac{n!}{(n-k)! \cdot k!}
	\end{equation*}
	
	\pause
	
	\texttt{bincoeff :: Int -> Int -> Int}
	
	\pause \bigskip
	
	\begin{lstlisting}[style=bg, firstnumber=5]
fac :: Int -> Int
fac n
	| n < 1     = 1
	| otherwise = n * fac (n-1)
	\end{lstlisting}

	\pause
	
	\begin{lstlisting}[style=bg]
bincoeff :: Int -> Int -> Int
bincoeff n k = fac n `div` (fac (n-k) * fac k)
	\end{lstlisting}
	
	\begin{alertblock}{Division}
		Der Operator \texttt{/} ist nur für Typen definiert, die gebrochene Zahlen darstellen können; \texttt{div} liefert dagegen die \texttt{Int}-Division.
	\end{alertblock}
\end{frame}
	%
\section{Aufgabe 2 \\ \textit{\normalsize Listen}}
	
	
	\begin{frame}[fragile] \frametitle{Listen}
		\begin{description}
			\item[Listen] Wenn \texttt{a} ein Typ ist, dann bezeichnet \texttt{[a]} den Typ ``Liste mit Elementen vom Typ \texttt{a}'', insbesondere haben alle Elemente einer Liste den gleichen Typ
			\bigskip \pause
			\item[cons-Operator `` \texttt{:} '']  ~\\ 
			Trennung von \textit{head} und \textit{tail} einer Liste \\
			\texttt{[x1, x2, x3, x4, x5] = x1 : [x2, x3, x4, x5]}
			\bigskip \pause
			\item[Verkettungsoperator `` \texttt{++} ''] ~\\ 
			Verkettung zweier Listen gleichen Typs \\
			\texttt{[x1, x2] ++ [x3, x4, x5] = [x1, x2, x3, x4, x5]}
		\end{description}
	\end{frame}


\begin{frame}[t, fragile] \frametitle{Aufgabe 2 -- Teil (a)}
	\textbf{Produkt einer Liste}
	
	\texttt{prod :: [Int] -> Int}
	
	\pause \bigskip
	
	\begin{lstlisting}[style=bg]
prod :: [Int] -> Int
prod []     = 1
prod (x:xs) = x * prod xs
	\end{lstlisting}
\end{frame}

\begin{frame}[t, fragile] \frametitle{Aufgabe 2 -- Teil (b)}
	\textbf{Umkehrung einer Liste}
	
	\texttt{rev :: [Int] -> [Int]}
	
	\pause \bigskip
	
	\begin{lstlisting}[style=bg]
rev :: [Int] -> [Int]
rev []     = []
rev (x:xs) = rev xs ++ [x]
	\end{lstlisting}

	\begin{alertblock}{WICHTIG}
		\begin{itemize}
			\item \texttt{Element : [Liste]}
			\item \texttt{[Liste] ++ [Liste]}
		\end{itemize}
	\end{alertblock}
\end{frame}

\begin{frame}[t, fragile] \frametitle{Aufgabe 2 -- Teil (c)}
	\textbf{Elemente einer Liste löschen}
	
	\texttt{excl :: Int -> [Int] -> [Int]}
	
	\pause \bigskip
	
	\begin{lstlisting}[style=bg]
excl :: Int -> [Int] -> [Int]
excl _  []    = []
excl y (x:xs)
	| x == y    = excl y xs
	| otherwise = x : excl y xs
	\end{lstlisting}
\end{frame}

\begin{frame}[t, fragile] \frametitle{Aufgabe 2 -- Teil (d)}
	\small 
	\textbf{Sortierung einer Liste prüfen}
	
	\texttt{isOrd :: [Int] -> Bool}
	
	\pause \bigskip
	\begin{lstlisting}[style=bg]
isOrd :: [Int] -> Bool
isOrd []  = True
isOrd [x] = True
isOrd (x:y:xs)
	| x <= y    = isOrd (y:xs)
	| otherwise = False
	\end{lstlisting}
	\pause
	\begin{lstlisting}[style=bg]
isOrd' :: [Int] -> Bool
isOrd' []       = True
isOrd' [x]      = True
isOrd' (x:y:xs) = x <= y && isOrd' (y:xs)
	\end{lstlisting}
\end{frame}

\begin{frame}[t, fragile] \frametitle{Aufgabe 2 -- Teil (e)}
	\small
	\textbf{sortiertes Zusammenfügen zweier (sortierten) Listen}
	
	\texttt{merge :: [Int] -> [Int] -> [Int]}
	
	\pause \bigskip
	\begin{lstlisting}[style=bg]
merge :: [Int] -> [Int] -> [Int]
merge [] ys = ys
merge xs [] = xs
merge (x:xs) (y:ys)
	| x < y     = x : merge xs (y:ys)
	| otherwise = y : merge (x:xs) ys
	\end{lstlisting}
	\pause
	Wir können Listen auch ``benennen'' --- Rekursionsfall:
	\begin{lstlisting}[style=bg]
merge xxs@(x:xs) yys@(y:ys)
	| x < y     = x : merge xs yys
	| otherwise = y : merge xxs ys
	\end{lstlisting}
\end{frame}

\begin{frame}[t, fragile] \frametitle{Aufgabe 2 -- Teil (f)}
	\small
	\textbf{(unendliche) Liste der Fibonacci-Zahlen}
	
	\texttt{fibs :: [Int]}
	
	\pause \bigskip
	\begin{lstlisting}[style=bg]
fib :: Int -> Int
fib 0 = 1
fib 1 = 1
fib n = fib (n-1) + fib (n-2)

fibs :: [Int]
fibs = fibAppend 0
	where fibAppend x = fib x : fibAppend (x+1)
	\end{lstlisting}
	\pause
	\begin{lstlisting}[style=bg]
fibs :: [Int]
fibs = fibs' 0 1
	where fibs' n m = n : fibs' m (n+m)
	\end{lstlisting}

	\textbf{Hinweis:} \texttt{take 7 fibs} liefert die ersten 7 Fibonacci-Zahlen
\end{frame}

%%%%%%%%%%%%%%%%%%%%%%%%%%%%%%%%%%%%%%%%%%%%%%%%%%%%%%%%%%%%%%%%%%%%%%%%%%%%%%%%%%%%

\begin{frame} \frametitle{Ende}
	\centering
	\textbf{Fragen?}
\end{frame}

\end{document}